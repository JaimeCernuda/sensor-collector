
\Section{Temporal Foundation for Bounded Uncertainty Synchronization}

Scientific workflows increasingly involve distributed data generation from heterogeneous sources (HPC compute nodes, edge instruments, monitoring services, external collaborators), each with independent system clocks that drift at different rates. Without temporal coordination, correlating log events across these sources becomes impossible or requires expensive hardware synchronization infrastructure. Achieving precise synchronization has historically required specialized hardware establishing accuracy ceilings while highlighting accessibility gaps. The White Rabbit Project~\cite{whiterabbit} achieves sub-nanosecond accuracy through Synchronous Ethernet and custom switches, while Google's TrueTime~\cite{spanner_osdi2012_truetime_gps_atomic} achieves 1-7ms uncertainty bounds using GPS receivers and atomic clocks, demonstrating that bounded uncertainty enables distributed algorithms but requiring expensive infrastructure impractical for edge devices and heterogeneous computing environments. Meta's PTP deployment~\cite{meta_ptp_time_appliances} demonstrates the substantial engineering investment required even for well-resourced organizations. Software approaches attempt to bridge this gap: the Huygens algorithm~\cite{huygens_nsdi2018_geng_svm_nanosec} applies support vector machines to filter network jitter achieving tens-of-nanoseconds accuracy within datacenters but operating reactively (measuring current error and correcting) rather than predictively. Machine learning has revolutionized hardware timing component prediction: LSTM models improve TCXO temperature compensation by 51\%~\cite{su_lstm_tcxo_transfer_learning} and satellite atomic clock prediction by 72\%~\cite{he_bds3_lstm_72pct_arima}. However, this work focuses on per-device calibration of individual components rather than end-to-end system clock behavior across heterogeneous hardware. Remarkably, despite extensive ML applications to distributed systems resource management (Google's Autopilot~\cite{autopilot_google_48pct_fleet_lstm}, Alibaba's DeepScaler~\cite{deepscaling_alibaba_30k_cores_gnn}), little published work applies these techniques to clock synchronization, likely due to disciplinary boundaries between clock research (favoring analytical models) and systems ML (focusing on resource management). No system provides predictive, zero-shot clock synchronization with quantified uncertainty bounds across commodity heterogeneous hardware. This temporal layer is essential for the other FlowForge components: HFlow's ByteFlow coordination benefits from ordered, windowed data delivery; Hades' in-transit operators require temporal context for stateful computation; and HStream's streaming engine depends on bounded temporal guarantees for windowing and event correlation.

In this work, FlowForge addresses Challenge 1's temporal dimension (enabling decentralized synchronization for log unification across distributed sources) by demonstrating that software-defined approaches can bridge the precision-accessibility gap in clock synchronization through three fundamental contributions:

\textbf{Time series foundation models as reliable clock predictors.} We establish that zero-shot time series forecasting models successfully predict system clock behavior without per-device calibration. FlowForge achieves 0.604 ms mean absolute error (Figure~\ref{fig:longterm_node2}), demonstrating sub-millisecond precision with sparse NTP measurements and 2.5-3x better results than state-of-the-art NTP synchronizers. As shown in Figure~\ref{fig:design_comparison}, the dual-model architecture balancing short-term responsiveness with long-term stability proves fundamental for capturing both transient variations and systematic drift patterns. This transforms clock synchronization from reactive measurement-correction to predictive temporal modeling.

\textbf{Effective uncertainty quantification enabling distributed algorithms.} FlowForge provides probabilistic uncertainty bounds through foundation model confidence intervals, achieving well-calibrated predictions that bound 95\% of measurements (Figure~\ref{subfig:longterm_node1_coverage}). This uncertainty quantification creates a pathway toward TrueTime-like algorithms using only NTP infrastructure, enabling log coordination algorithms that tolerate moderate uncertainty if that uncertainty is quantified and bounded.

\textbf{Production-ready platform with long-term operational capability.} The complete software-defined clock system demonstrates practical viability through 10+ hour continuous operation across diverse hardware platforms without specialized timing infrastructure (Figures~\ref{fig:longterm_node1} and~\ref{fig:longterm_node2}) by providing robust error handling through defensive mechanisms that address both external measurement anomalies and internal prediction failures.

Through this work, FlowForge's bounded clocks enable temporal operations essential to the platform. Global event ordering across distributed sources allows HFlow's ByteFlows to coordinate with temporal coherence, ensuring that data movements from multiple sources can be sequenced correctly. Time-windowing enables HStream to correlate events from multiple streams despite temporal drift, supporting operations like multi-stream joins that require co-temporal data. Temporal alignment allows Hades' operators to compute on temporally coherent windows without requiring centralized synchronization, enabling distributed in-transit analysis. By providing accessible, quantified temporal semantics, FlowForge transforms the other components from isolated systems managing their own data/compute/forwarding concerns into a cohesive platform capable of managing the complete temporal, spatial, and computational dimensions of scientific log data.

