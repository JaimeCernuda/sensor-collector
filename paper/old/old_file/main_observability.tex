\documentclass[conference]{IEEEtran}
\IEEEoverridecommandlockouts

\usepackage{graphicx}
\usepackage{booktabs}
\usepackage{amsmath}
\usepackage{algorithm}
\usepackage{algpseudocode}
\usepackage{subcaption}
\usepackage{multirow}
\usepackage{hyperref}
\usepackage{xcolor}
\usepackage{colortbl}
\usepackage{pifont}

\hypersetup{
    colorlinks=true,
    linkcolor=blue,
    urlcolor=blue,
    citecolor=blue
}

% Table markers
\newcommand{\cmark}{\cellcolor{green!20}{\ding{51}}}
\newcommand{\xmark}{\cellcolor{red!20}{\ding{55}}}
\newcommand{\nmark}{\cellcolor{yellow!30}N/A}

\begin{document}

\title{ChronoTick: Precise Temporal Semantics for Distributed Systems Observability}

\author{
\IEEEauthorblockN{1\textsuperscript{st} Given Name Surname}
\IEEEauthorblockA{\textit{dept. name of organization (of Aff.)} \\
\textit{name of organization (of Aff.)}\\
City, Country \\
email address or ORCID}
\and
\IEEEauthorblockN{2\textsuperscript{nd} Given Name Surname}
\IEEEauthorblockA{\textit{dept. name of organization (of Aff.)} \\
\textit{name of organization (of Aff.)}\\
City, Country \\
email address or ORCID}
\and
\IEEEauthorblockN{3\textsuperscript{rd} Given Name Surname}
\IEEEauthorblockA{\textit{dept. name of organization (of Aff.)} \\
\textit{name of organization (of Aff.)}\\
City, Country \\
email address or ORCID}
}

\maketitle

\begin{abstract}
Distributed systems observability relies on correlating events across independent services, but accurate correlation requires temporal agreement that commodity infrastructure cannot guarantee. Current observability stacks (OpenTelemetry, Jaeger, Zipkin) depend on NTP with 1--10\,ms uncertainty---insufficient for sub-millisecond event ordering in microservice architectures. Logical clocks provide causal ordering but lose real-time semantics essential for cross-system correlation and human-readable timelines. We present ChronoTick, a lightweight daemon that provides precise, bounded timestamps to observability infrastructure through ML-driven clock prediction. ChronoTick applies zero-shot time series foundation models to predict clock drift between NTP measurements, achieving 0.604\,ms mean absolute error with 94.9\% uncertainty coverage. Its lock-free shared memory interface adds only 2\,$\mu$s overhead per timestamp---negligible compared to typical instrumentation costs. A single daemon instance serves all services on a node, requiring no application modification beyond replacing system clock calls. By providing bounded uncertainty (``this event occurred at $T \pm \epsilon$''), ChronoTick enables principled ordering decisions: events are correctly ordered when their uncertainty intervals do not overlap. We evaluate across heterogeneous platforms and discuss implications for trace ordering accuracy and log correlation in production observability pipelines.
\end{abstract}

\begin{IEEEkeywords}
distributed observability, temporal semantics, clock synchronization, event ordering, time series foundation models
\end{IEEEkeywords}

\section{Introduction}

Every commodity server relies on a cheap, uncompensated quartz crystal oscillator (XO) whose frequency varies $\pm$20--50\,ppm with temperature. A 10\,$^{\circ}$C temperature swing from CPU load produces ${\sim}$10\,$\mu$s/s of uncorrected drift. Over minutes of holdover between NTP corrections, this accumulates to tens or hundreds of microseconds of clock error. NTP~\cite{ntp} and PTP~\cite{ptp} correct drift periodically, but between corrections the clock free-runs. Temperature-compensated crystal oscillators (TCXOs) reduce this sensitivity by $10{-}100\times$ through analog compensation circuits, but they cost \$5--50 per unit and require board-level integration unavailable in commodity servers.

The question driving this work is: \textit{how far can software-only correction using modern ML and foundation time series models push oscillator compensation on commodity hardware?}

The single most relevant prior work is Graham~\cite{graham_nsdi22}, which demonstrated that reading commodity DIMM temperature sensors and fitting cubic polynomials to the temperature-frequency relationship reduces drift by 2000$\times$ (200\,ppm to 100\,ppb), effectively building a software-defined TCXO. Graham's key finding was that the DIMM sensor serves as a better crystal proxy than the CPU die temperature due to lower thermal lag, and that each crystal has a unique frequency-temperature curve due to manufacturing variance in AT-cut angle. This landmark result established that commodity \texttt{hwmon} sensors contain sufficient signal for software-defined clock compensation.

However, Graham's approach has seven significant limitations that define our research gap:
\begin{enumerate}
    \item \textbf{Cubic polynomial only}: no ML, no ensemble methods, no neural approaches---limited to a fixed parametric model that cannot capture complex nonlinear dynamics.
    \item \textbf{48-hour per-device offline calibration}: each server requires a dedicated multi-day calibration run before compensation can begin, impractical at datacenter scale.
    \item \textbf{No hysteresis modeling}: crystals drift differently during heating versus cooling (${\sim}$50--200\,ppb hysteresis), which a static polynomial cannot capture.
    \item \textbf{No aging adaptation}: crystal oscillators age 3--5\,ppm/year; a statically calibrated model becomes stale without continuous adaptation.
    \item \textbf{Single sensor}: uses only DIMM temperature, ignoring dozens of available thermal zones, CPU frequency steps, interrupt rates, and other correlated signals.
    \item \textbf{Not open source}: code was never released, preventing reproduction or extension.
    \item \textbf{No uncertainty bounds}: provides point corrections only, with no confidence intervals for downstream algorithms to reason about correction quality.
\end{enumerate}

Meanwhile, \texttt{chrony}'s \texttt{tempcomp} directive already provides quadratic polynomial compensation from a single \texttt{hwmon} sensor, achieving a 3.54$\times$ improvement (offset standard deviation from 6.3\,$\mu$s to 1.78\,$\mu$s)~\cite{ntp_performance_chrony_ntpd}. This establishes our minimum baseline to beat.

Recent advances in time series foundation models (TSFMs) present an opportunity to address all seven limitations simultaneously. Models including Chronos~\cite{Chronos}, TimesFM~\cite{TimesFM}, Moirai~\cite{MOIRAI}, and TTMs~\cite{TTMs} achieve competitive forecasting accuracy across diverse domains through zero-shot inference---requiring no per-device training or fine-tuning. Critically, these models produce probabilistic forecasts through multi-quantile outputs that directly encode prediction uncertainty, providing the confidence intervals Graham's approach lacks. The zero-shot capability directly addresses the calibration weakness: a pretrained model can predict drift without per-device training.

The intersection of \{ML beyond polynomials\} + \{commodity Linux \texttt{hwmon} sensors\} + \{clock synchronization integration\} + \{open source\} + \{uncertainty bounds\} is \textbf{completely unoccupied} in the literature.

We present ChronoTick, a system that fills this gap through the following contributions:

\begin{enumerate}
    \item \textbf{Multivariate ML drift prediction}: ChronoTick leverages not just temperature but CPU frequency, system load, and multiple thermal zones---capturing environmental dynamics that single-sensor polynomial approaches miss.
    \item \textbf{Foundation time series models for clock drift}: We are the first to apply zero-shot TSFMs to oscillator drift prediction. Zero-shot inference eliminates Graham's 48-hour per-device calibration entirely.
    \item \textbf{Online adaptive correction}: A retrospective correction mechanism continuously refines the prediction dataset, enabling adaptation to crystal aging and environmental changes without manual recalibration.
    \item \textbf{Open-source system}: ChronoTick is the first open-source implementation of ML-based clock compensation, shipping the full pipeline from data collection through model inference to real-time correction.
    \item \textbf{Uncertainty-aware corrections}: Quantile-based uncertainty extraction from foundation model outputs provides calibrated confidence intervals achieving 94.9\% coverage, enabling downstream algorithms to reason about correction quality.
\end{enumerate}

ChronoTick achieves 0.604\,ms MAE across heterogeneous platforms without per-device training, maintaining 2.5--3$\times$ better temporal consistency than commodity NTP while providing TrueTime-equivalent probabilistic guarantees on commodity hardware.

\section{Related Work}

\subsection{Clock Compensation}

The most direct prior work is chrony's \texttt{tempcomp} directive~\cite{ntp_performance_chrony_ntpd}, which applies quadratic polynomial compensation from a single \texttt{hwmon} sensor. Measurements by IJS.si demonstrated a 3.54$\times$ improvement in offset standard deviation (6.3\,$\mu$s to 1.78\,$\mu$s), establishing a practical baseline for software-defined compensation on commodity hardware.

Graham~\cite{graham_nsdi22} demonstrated that reading commodity DIMM temperature sensors and fitting cubic polynomials to the temperature-frequency relationship reduces drift by 2000$\times$ (200\,ppm to 100\,ppb). Key findings include that DIMM sensors serve as better crystal proxies than CPU die temperature (lower thermal lag), and that each crystal requires individual calibration due to manufacturing variance in AT-cut angle. However, Graham's approach is limited to a fixed parametric model, requires 48-hour per-device calibration, ignores hysteresis and aging effects, uses only a single sensor, provides no uncertainty bounds, and was never released as open source.

\subsection{ML for Hardware Timing}

Machine learning has been applied to individual timing component prediction. Su et al.~\cite{su_lstm_tcxo_transfer_learning} apply LSTM models to improve TCXO temperature compensation by 51\% through transfer learning across oscillator units. He et al.~\cite{he_bds3_lstm_72pct_arima} demonstrate 72\% improvement over ARIMA for BDS-3 satellite atomic clock prediction using LSTM networks. These works validate that neural approaches capture timing dynamics beyond traditional analytical models, but focus on per-device calibration of individual components rather than end-to-end system clock behavior across heterogeneous hardware.

The Huygens algorithm~\cite{huygens_nsdi2018_geng_svm_nanosec} applies support vector machines to filter network jitter, achieving tens-of-nanoseconds accuracy within datacenters. However, Huygens operates reactively (measuring and correcting current error) rather than predictively, requires PTP-capable NIC hardware timestamps, and was never released as open source. It was commercialized as Clockwork Systems.

\subsection{Foundation Time Series Models}

Time series foundation models (TSFMs) represent a paradigm shift from per-task model training to zero-shot generalization. Chronos~\cite{Chronos} converts time series into discrete tokens using scaling and quantization, achieving strong zero-shot performance through pretraining on diverse datasets. TimesFM~\cite{TimesFM} is a decoder-only foundation model pretrained on 100 billion real-world time points, predicting patches autoregressively for efficient long-horizon forecasting. Moirai~\cite{MOIRAI} addresses time series heterogeneity through multi-patch layers and flexible attention for multivariate inputs. TTMs~\cite{TTMs} is the first pretrained lightweight model explicitly capturing cross-channel dependencies and external covariates.

Critically for clock compensation, these models produce probabilistic forecasts through multi-quantile outputs that directly encode prediction uncertainty---a capability absent from polynomial compensation approaches. The zero-shot property eliminates per-device calibration, and the ability to consume multivariate inputs enables leveraging the full set of available \texttt{hwmon} sensors rather than a single temperature reading.

\subsection{Software-Defined Timing}

AWS ClockBound provides software clock error bounding via chrony integration but relies on static error models without ML prediction. Google's TrueTime~\cite{spanner_osdi2012_truetime_gps_atomic} achieves bounded uncertainty through GPS receivers and atomic clocks---hardware costing \$50,000+ per datacenter. Meta's Simplified PTP (SPTP)~\cite{meta_ptp_time_appliances} reduces PTP complexity but still requires specialized network infrastructure. CockroachDB~\cite{cockroachdb_hlc_implementation} and YugabyteDB~\cite{yugabytedb_hlc_distributed} employ Hybrid Logical Clocks for causal consistency without specialized hardware but sacrifice real-time semantics.

None of these systems provides ML-based drift prediction from multivariate sensor data on commodity hardware. ChronoTick occupies the unoccupied intersection of ML drift prediction, commodity sensors, uncertainty quantification, and open-source availability.


\section{ChronoTick Design}

ChronoTick is a predictive, AI-driven time synchronization system architected specifically for decentralized time coordination. Unlike reactive synchronization approaches that correct drift after detection, ChronoTick leverages foundation models to predict clock behavior patterns before they manifest, achieving both higher precision and lower latency through intelligent anticipation of system dynamics. The system integrates transparently into existing programs with a simple and highly performant API.

ChronoTick bridges the precision gap between NTP (1-10 ms) and PTP (sub-microsecond) through software-defined time, paralleling how Software-Defined Networking replaced specialized routing hardware with programmable control planes. The key insight is increasing the effective synchronization frequency from sparse NTP measurements (every 1-10 minutes in typical deployments) to continuous predictions (every second), analogous to how frame generation increases frame rate through interpolation. By predicting clock behavior between NTP measurements rather than waiting for the next synchronization event, ChronoTick achieves sub-millisecond precision using only commodity infrastructure: no specialized NICs, GPS receivers, or atomic clocks required.

\subsection{Architecture}

\begin{figure}[h]
    \centering
    \includegraphics[width=\linewidth]{figures/chronotick_arch.pdf}
    \caption{ChronoTick predictive synchronization architecture.}
    \label{fig:arch_htick}
\end{figure}

The architecture of ChronoTick, shown in Figure~\ref{fig:arch_htick}, implements a five-layer predictive synchronization stack: the \textit{Client Interface}, \textit{Scheduler}, \textit{Prediction Layer}, \textit{Retrospective Correction}, and \textit{Data Collection}. This design prioritizes response latency through predictive scheduling that decouples model inference latency from client request response times, enabling the system to serve as a viable replacement for operating system time calls across diverse deployment scenarios.

The \textbf{Client Interface} addresses a fundamental scalability challenge: operating system time calls complete in hundreds of nanoseconds, yet ChronoTick must serve corrected timestamps to arbitrarily many concurrent clients at comparable latencies without blocking on inference computations. Applications requiring corrected timestamps---databases, distributed coordinators, monitoring systems---all require synchronized timestamps simultaneously without contention. A single ChronoTick daemon must serve all these concurrent clients at microsecond-scale latencies. The system achieves this through a daemon architecture with lock-free shared memory, enabling unlimited concurrent readers to access pre-computed corrections without mutual exclusion overhead. This separation between correction computation and correction delivery enables ChronoTick to be a viable drop-in replacement for system time calls.

The \textbf{Scheduler} addresses the orders-of-magnitude mismatch between model inference timescales and client query latencies by pre-computing forecasts before clients request them. Foundation model inference operates at tens of milliseconds while client applications demand sub-millisecond responses, creating an architectural tension. By scheduling predictions ahead of anticipated need and maintaining temporal caches of forecasts, the system completely eliminates inference latency from the critical path of client queries.

At the heart of the system, the \textbf{Prediction Layer} employs a dual-model architecture to balance temporal responsiveness with prediction stability. A short-horizon model with frequent updates captures transient variations while a long-horizon model with infrequent updates provides stable trend estimation. These complementary forecasting systems leverage time series foundation models for zero-shot prediction, producing both point estimates and uncertainty bounds through probabilistic quantile outputs. Predictions combine via inverse variance weighted fusion, automatically adapting to model confidence levels to minimize uncertainty across forecast windows.

The \textbf{Retrospective Correction} maintains dataset quality by validating predictions against external synchronization measurements. When NTP measurements arrive, the system identifies prediction errors by comparing forecasts to ground truth. The system then corrects the historical dataset by interpolating between NTP measurements, replacing erroneous predictions with accurate values. This corrected dataset becomes the context foundation for subsequent predictions, enabling the autoregressive models to learn from mistakes and adapt to changing system dynamics.

The \textbf{Data Collection} forms the foundation, harvesting timing data from network protocols and system performance metrics with dedicated CPU affinity to minimize jitter. Progressive refinement through statistical outlier filtering and temporal smoothing transforms noisy measurements into clean context data suitable for foundation model consumption. System exogenous variables including temperature, processor frequency, and load enable models to distinguish predictable environmental effects from genuine timing anomalies---a key differentiator versus Graham's single DIMM sensor approach~\cite{graham_nsdi22}.

This architectural flow enables ChronoTick to achieve demanding precision requirements while operating entirely on commodity hardware through software-only prediction and correction mechanisms.

\subsection{Client Interface}

For ChronoTick to serve as a practical replacement for operating system time calls, the client interface must match the performance characteristics applications expect from system services: sub-microsecond latency, unlimited concurrent access, and zero contention between readers. ChronoTick addresses this scalability challenge through a daemon architecture that pre-computes corrections in a dedicated process while exposing results via lock-free shared memory accessible to arbitrarily many concurrent clients.

The daemon process operates with dedicated CPU affinity (a non-administrative operation for self-owned processes), isolated from application workloads to ensure consistent prediction computation without interference from client activity. The daemon maintains continuous operation across client lifecycle events, surviving individual application failures while providing corrections to all active clients regardless of their execution state or resource consumption patterns.

Lock-free shared memory enables concurrent reads without mutual exclusion overhead. The system implements a single-writer-multiple-reader pattern where the daemon writes pre-computed corrections to cache-aligned shared memory while an unlimited number of clients read concurrently without locks, atomic operations, or coordination overhead. Sequence number validation detects torn reads during concurrent updates, enabling readers to retry on the rare occasions when writes interrupt reads. This lock-free design achieves read latencies measured in single microseconds, even in high parallelism scenarios, while maintaining correctness guarantees. The shared memory buffer contains not only correction values but comprehensive metadata including uncertainty bounds, drift rates for client-side interpolation, validity periods, and operational status.

\subsection{Multivariate Data Collection}

The data collection layer acquires timing measurements optimized for foundation model input requirements, collecting synchronized timing data alongside system performance metrics with controlled frequency during warmup and operational phases. Collection operates on the daemon CPU core with elevated priority to minimize measurement jitter, which proves critical for accurate drift pattern detection by forecasting models.

Unlike prior approaches that rely on a single temperature sensor~\cite{graham_nsdi22}, ChronoTick collects multivariate system telemetry: CPU temperature across multiple thermal zones, processor frequency (which steps discretely under load), system load, and additional \texttt{hwmon} sensors when available. This multivariate approach captures the full environmental context affecting oscillator behavior, including thermal transients from CPU load changes, frequency scaling effects on timing circuits, and cross-correlated system dynamics that single-sensor polynomial approaches miss.

ChronoTick implements NTP measurement querying multiple authoritative time servers using standard synchronization protocols. Each measurement exchange computes clock offset using the four-timestamp calculation: $o = ((t_2 - t_1) + (t_3 - t_4)) / 2$, where $t_1$ and $t_4$ represent local timestamps while $t_2$ and $t_3$ represent server timestamps. Network delay and measurement uncertainty follow from $d = (t_4 - t_1) - (t_3 - t_2)$ and $\sigma_{\text{measurement}} = \max(d/2, \sigma_{\text{server}})$ respectively. Multiple servers are queried with outlier rejection using median absolute deviation filtering to eliminate anomalous measurements caused by network congestion. The median offset with mean round-trip delay provides improved measurement quality, reducing uncertainty.

The system operates through two distinct phases addressing the cold-start challenge: foundation models require historical context to generate meaningful forecasts, yet no such context exists during initial deployment. During warmup, the system operates in conservative mode collecting network time measurements at elevated frequency (every 5 seconds) while delivering corrections derived directly from these measurements without model predictions. This bootstrap period accumulates sufficient measurement history for models to establish baseline clock behavior patterns while avoiding premature predictions from models operating with insufficient context. After quality gates confirm adequate dataset accumulation (typically 60 seconds yielding 12 measurements), the system transitions to normal operation, activating predictive scheduling and retrospective correction mechanisms.

\subsection{Scheduler}

The predictive scheduler addresses a fundamental architectural challenge in applying foundation models to real-time coordination: model inference operates on timescales of tens of milliseconds, while applications demand sub-millisecond response latency for temporal queries. Computing corrections on-demand inherently couples inference latency to response latency making them incompatible. ChronoTick resolves this tension through predictive scheduling that completely decouples prediction computation from prediction delivery, pre-computing forecasts before clients request them and serving cached results with microsecond-scale lookup latency.

\textbf{Insight: Predictive Scheduling.}~\textit{Foundation model inference (45-100ms) is orders of magnitude slower than required response latency (<1ms). By scheduling prediction tasks ahead of when results will be requested and maintaining temporal caches, the system transforms ML latency characteristics from blocking operations to asynchronous background tasks, enabling foundation models to serve real-time systems.}

The scheduling architecture operates through self-rescheduling prediction tasks that maintain continuous forecast availability across future time horizons. Each forecasting model schedules its next prediction cycle upon completing current inference, creating autonomous prediction loops that populate the forecast cache without external coordination. The short-horizon model schedules frequent predictions within the bounds of their forecasting window, ensuring that forecasts are always available while minimizing cache staleness. The long-horizon model schedules predictions infrequently, but further in advance to account for the longer inference times. This self-scheduling design proves robust to inference time variations and model failures, as each prediction task independently reschedules itself rather than depending on centralized scheduling logic that could create single points of failure.

The scheduler also manages the system cache containing the set of predictions from the models. The cache maintains a rolling window of predictions centered on current time, evicting forecasts for temporal points that have already passed while retaining those for upcoming time points where clients are likely to request corrected timestamps. Cache hits deliver corrections with sub-millisecond latency through simple temporal lookup, while cache misses trigger fallback mechanisms that compute on-demand predictions at the cost of increased response latency.

\subsection{Prediction Layer}

ChronoTick implements a dual-model forecasting architecture leveraging time series foundation models to predict clock drift patterns before they occur, enabling proactive rather than reactive synchronization. The architecture balances temporal responsiveness with prediction stability through complementary forecasting models operating at different timescales, update frequencies, and prediction horizons.

The short-horizon model operates with frequent update intervals of short prediction windows, utilizes limited historical context enriched with system metrics to capture rapid fluctuations and transient anomalies while maintaining low inference latency. This model provides immediate corrections, prioritizing responsiveness over long-term trend accuracy. Conversely, the long-horizon model updates with less frequency and extended prediction windows using substantial historical context, providing smooth and stable drift trend estimation resistant to measurement noise and short-term variations. This model accepts longer inference times given the infrequent update cycle, prioritizing prediction stability and uncertainty quantification over responsiveness.

\textbf{Insight: Dual-Model Temporal Decomposition.}~\textit{Employ two complementary forecasting models: a short-horizon model with frequent updates capturing system metrics and rapid variations, and a long-horizon model with infrequent updates providing stable trend estimation. This captures both transient thermal effects and long-term patterns (diurnal cycles, aging).}

Required foundation models for ChronoTick must produce probabilistic forecasts through multi-quantile outputs that directly encode prediction uncertainty. Rather than requiring separate uncertainty estimation procedures, the models output predictions at multiple probability levels simultaneously, enabling direct extraction of confidence intervals.

\subsubsection{Prediction Inputs and Outputs}

Before describing how ChronoTick corrects system timestamps, we clarify the inputs and outputs of the prediction layer. The prediction layer consumes historical time series data to forecast future clock behavior, producing predicted skew and drift values with uncertainty bounds.

\begin{table}[h]
\centering
\small
\caption{Prediction Layer Inputs}
\label{tab:prediction_inputs}
\begin{tabular}{@{}lp{6.5cm}l@{}}
\toprule
\textbf{Input} & \textbf{Description} & \textbf{Format} \\ \midrule
Historical skew & Time series of normalized skew measurements & ms sequence \\
Historical drift & Time series of normalized drift rate measurements & ppm sequence \\
CPU temperature & Processor thermal state & Celsius \\
CPU frequency & Current processor clock speed & GHz \\
System load & CPU utilization percentage & percent \\
\bottomrule
\end{tabular}
\end{table}

\begin{table}[h]
\centering
\small
\caption{Prediction Layer Outputs}
\label{tab:prediction_outputs}
\begin{tabular}{@{}lp{6.5cm}ll@{}}
\toprule
\textbf{Output} & \textbf{Description} & \textbf{Symbol} & \textbf{Units} \\ \midrule
Predicted skew & Instantaneous clock offset from true time & $\hat{s}$ & ms \\
Predicted drift & Rate of clock divergence & $\hat{r}$ & ppm \\
Skew uncertainty & Confidence bound on predicted skew & $\sigma_s$ & ms \\
Drift uncertainty & Confidence bound on predicted drift & $\sigma_r$ & ppm \\
\bottomrule
\end{tabular}
\end{table}

The predicted \textit{skew} $\hat{s}$ represents the instantaneous offset between the system clock and true time at the moment of prediction. The predicted \textit{drift} $\hat{r}$ represents the rate at which the system clock diverges from true time, measured in parts per million (ppm); for example, 10 ppm means the clock gains or loses 10 microseconds per second. These are distinct quantities: skew is corrected by adding an offset to the system clock, while drift is used to compensate for time elapsed since the nearest prediction. The foundation models predict both quantities simultaneously through multi-horizon forecasting, with uncertainty bounds $\sigma_s$ and $\sigma_r$ extracted directly from quantile outputs.

\subsubsection{Quantile-Based Uncertainty Extraction}

\textbf{Insight: Quantile-Based Uncertainty Quantification.}~\textit{Extract uncertainty estimates directly from foundation model quantile outputs rather than training separate uncertainty estimation models.}

\begin{algorithm}[H]
\caption{Quantile-Based Uncertainty Extraction}
\label{algo:quantile_uncertainty}
\begin{algorithmic}[1]
\Require Model quantile predictions $\{q_p : p \in [0.1, 0.2, \ldots, 0.9]\}$, Confidence level $\alpha$ (default 0.8)
\Ensure Point prediction $\hat{y}$, Uncertainty estimate $\sigma$
\State $\hat{y} \gets q_{0.5}$ \Comment{Median as point prediction}
\State $q_{\text{low}} \gets q_{0.1}$ \Comment{10th percentile}
\State $q_{\text{high}} \gets q_{0.9}$ \Comment{90th percentile}
\State $\sigma \gets \frac{q_{\text{high}} - q_{\text{low}}}{2.56}$ \Comment{$2.56 \approx 2 \times \Phi^{-1}(0.9)$ for normal distribution}
\State \Return $\hat{y}, \sigma$
\end{algorithmic}
\end{algorithm}

The conversion factor 2.56 arises from the relationship between quantile spreads and standard deviations under the Gaussian assumption: the interval from the 10th to 90th percentile spans approximately $2 \times 1.28\sigma = 2.56\sigma$ for a normal distribution, where $\Phi^{-1}(0.9) \approx 1.28$ is the inverse cumulative distribution function. This extraction provides uncertainty estimates directly from the model's probabilistic output without requiring separate calibration procedures, though the system can optionally apply empirical correction factors when the Gaussian assumption proves inaccurate for specific deployment environments.

\subsubsection{Inverse Variance Fusion}

The system combines predictions using mathematically optimal inverse variance weighted fusion that automatically adapts to model confidence levels. When both models provide predictions with uncertainty estimates represented by standard deviations $\sigma_{\text{short}}$ and $\sigma_{\text{long}}$, the fusion weights derive from inverse variance:

\begin{algorithm}[H]
\caption{Inverse Variance Weighted Prediction Fusion}
\label{algo:fusion}
\begin{algorithmic}[1]
\Require Short-term prediction $\hat{y}_{\text{short}}$ with uncertainty $\sigma_{\text{short}}$, Long-term prediction $\hat{y}_{\text{long}}$ with uncertainty $\sigma_{\text{long}}$
\Ensure Fused prediction $\hat{y}_{\text{fused}}$, Fused uncertainty $\sigma_{\text{fused}}$
\State $w_{\text{short}} \gets \frac{1/\sigma_{\text{short}}^2}{1/\sigma_{\text{short}}^2 + 1/\sigma_{\text{long}}^2}$ \Comment{Compute inverse variance weights}
\State $w_{\text{long}} \gets \frac{1/\sigma_{\text{long}}^2}{1/\sigma_{\text{short}}^2 + 1/\sigma_{\text{long}}^2}$
\State $\hat{y}_{\text{fused}} \gets w_{\text{short}} \cdot \hat{y}_{\text{short}} + w_{\text{long}} \cdot \hat{y}_{\text{long}}$ \Comment{Weighted combination}
\State $\sigma_{\text{fused}} \gets \frac{1}{\sqrt{1/\sigma_{\text{short}}^2 + 1/\sigma_{\text{long}}^2}}$ \Comment{Optimal uncertainty reduction}
\State \Return $\hat{y}_{\text{fused}}, \sigma_{\text{fused}}$
\end{algorithmic}
\end{algorithm}

This fusion strategy is provably optimal in minimizing the variance of the combined prediction when both models are unbiased estimators of the true value. The fused uncertainty $\sigma_{\text{fused}}$ is always smaller than either individual model uncertainty, achieving variance reduction through the statistical principle that combining independent estimates reduces total uncertainty. When one model reports much higher uncertainty than the other, the fusion automatically weights the more confident prediction more heavily, enabling graceful degradation when individual models encounter difficulties while maintaining overall system precision.

\subsection{Time Correction Mechanism}

ChronoTick synthesizes corrected physical time by applying predicted skew and drift compensation to the system clock, providing temporal precision exceeding traditional reactive approaches.

\textbf{Insight: System Clock Correction with Predictive Skew.}~\textit{Correct the system clock by applying the predicted skew from the nearest prediction, then walk forward with elapsed time and predicted drift compensation has proven more accurate than walking from the previous NTP measurement.}

The correction engine computes timestamps by applying a two-step process. First, it corrects the current system clock using the nearest predicted skew $\hat{s}_{\text{nearest}}$ from the prediction cache. Second, it compensates for elapsed time since that prediction using the predicted drift rate $\hat{r}_{\text{drift}}$; this ``walking forward'' process accounts for how the system clock continues to diverge during the interval between the prediction time and the query time.

\begin{algorithm}[H]
\caption{System Clock Correction with Predictive Skew}
\label{algo:time_correction}
\begin{algorithmic}[1]
\Require Current system time $t_{\text{system}}$, Nearest predicted skew $\hat{s}_{\text{nearest}}$ at $t_{\text{nearest}}$, Predicted drift rate $\hat{r}_{\text{drift}}$ (ppm), Skew uncertainty $\sigma_s$, Drift uncertainty $\sigma_r$
\Ensure Corrected time $t_{\text{corrected}}$, Total uncertainty $\sigma_{\text{total}}$
\State $\Delta t \gets t_{\text{system}} - t_{\text{nearest}}$ \Comment{Elapsed since nearest prediction}
\State $t_{\text{corrected}} \gets t_{\text{system}} + \hat{s}_{\text{nearest}} + \hat{r}_{\text{drift}} \cdot \Delta t$ \Comment{Apply skew + drift compensation}
\State $\sigma_{\text{total}} \gets \sqrt{\sigma_s^2 + (\sigma_r \cdot \Delta t)^2}$ \Comment{Uncertainty propagation}
\State \Return $t_{\text{corrected}}, \sigma_{\text{total}}$
\end{algorithmic}
\end{algorithm}

This approach proved more successful in practice than NTP-anchored correction, which while accurate, exhibited lower precision due to the temporal distance between sparse NTP measurements and query times. Still, NTP-anchoring can be used if violations of the monotonicity of the clock are observed on the system.

Total uncertainty combines prediction uncertainty with temporal degradation growing with elapsed time since the nearest prediction. The uncertainty propagation formula $\sigma^2_{\text{total}} = \sigma^2_s + \sigma^2_r \cdot (\Delta t)^2$ captures two distinct error sources: the prediction uncertainty $\sigma_s$ from the model forecast, and accumulating drift uncertainty growing quadratically with elapsed time. Corrections apply with strict monotonicity guarantees ensuring temporal causality: $t_{\text{corrected}}(t_1) < t_{\text{corrected}}(t_2)$ for all $t_1 < t_2$.

\subsection{Retrospective Correction}

The retrospective correction module maintains prediction accuracy by integrating external synchronization measurements into the historical dataset, allowing new high-quality measurements to reveal previous prediction errors. The system must thus incorporate this information to improve future predictions while preserving temporal causality. ChronoTick implements retrospective correction algorithms spanning a design spectrum between causal consistency preservation and rapid accuracy improvement.

\textbf{Insight: Retrospective Dataset Correction.}~\textit{When new NTP measurements reveal prediction errors, correct the historical dataset retroactively rather than only adjusting future predictions, enabling foundation models to learn from mistakes through autoregressive training.}

Time series foundation models are autoregressive: they predict future values based on recent historical context. If the historical context contains uncorrected prediction errors, the model continues making similar mistakes. When high-confidence external synchronization occurs, the system must reconcile the discrepancy $\delta = s_{\text{true}} - \hat{s}$ between true measured skew $s_{\text{true}}$ and predicted skew $\hat{s}$ at time $t$. By replacing erroneous predictions with interpolated NTP ground truth, the corrected dataset teaches the model true clock behavior patterns. The production system employs backtracking correction, which provides the strongest learning signal by replacing the entire model context window with interpolated NTP measurements:

\begin{algorithm}[H]
\caption{Backtracking Retrospective Correction}
\label{algo:backtracking}
\begin{algorithmic}[1]
\Require Previous NTP skew $s_{\text{prev}}$ at time $t_{\text{prev}}$, Current NTP skew $s_{\text{curr}}$ at time $t_{\text{curr}}$, Historical timestamps $\{t_i\}_{i=1}^{n}$ in $(t_{\text{prev}}, t_{\text{curr}})$, Predicted skews $\{\hat{s}_{t_i}\}_{i=1}^n$, Context window size $w_{\text{context}}$
\Ensure Corrected skews $\{\hat{s}'_{t_i}\}_{i=1}^{n}$
\State $t_{\text{start}} \gets \max(t_{\text{prev}} - w_{\text{context}}, t_{\text{experiment\_start}})$ \Comment{Extend correction window}
\For{$i = 1$ to $n$}
    \If{$t_i \geq t_{\text{start}}$}
        \State $\alpha \gets \frac{t_i - t_{\text{prev}}}{t_{\text{curr}} - t_{\text{prev}}}$ \Comment{Linear interpolation}
        \State $\hat{s}'_{t_i} \gets s_{\text{prev}} + \alpha \cdot (s_{\text{curr}} - s_{\text{prev}})$
    \Else
        \State $\hat{s}'_{t_i} \gets \hat{s}_{t_i}$ \Comment{Preserve predictions outside window}
    \EndIf
\EndFor
\State \Return $\{\hat{s}'_{t_i}\}_{i=1}^{n}$
\end{algorithmic}
\end{algorithm}

The backtracking algorithm extends the correction window beyond the interval between consecutive NTP measurements to cover the full context window that foundation models consume for generating predictions. For a model with context window $w_{\text{context}} = 512$ measurements at 1 Hz sampling (512 seconds), the correction window extends 512 seconds backward from the current NTP measurement. All skew and drift predictions within this extended window are corrected with linear interpolation between bounding NTP anchor points, effectively transforming the dataset into what would have been observed if NTP measurements had been available continuously.

This eliminates systematic prediction biases from the training dataset: if the model consistently over-predicts or under-predicts drift, these errors are removed and replaced with ground truth interpolations. It ensures the entire context window fed to the model during the next prediction cycle contains NTP-aligned data rather than a mixture of accurate and erroneous predictions, preventing error propagation through the autoregressive mechanism. This creates a continuous improvement loop: predictions are generated, validated against NTP ground truth when measurements arrive, errors are corrected in the historical dataset, and improved predictions are generated using the corrected history as context. For oscillator drift prediction specifically, this mechanism addresses crystal aging and environmental changes without manual recalibration---the model continuously adapts to the current frequency-temperature relationship rather than relying on a stale calibration.

\subsection{Defense Mechanisms for Production Robustness}

ChronoTick implements multi-layer defensive validation ensuring robust operation despite noisy measurements and occasional model errors. These mechanisms operate at two levels: external defenses protecting against measurement anomalies, and internal defenses constraining model predictions to physically plausible ranges.

\textbf{External Defense Mechanisms:} To ensure the correct correction of the time data and avoid context poisoning of the model, progressive refinement through statistical outlier filtering and temporal smoothing transforms possibly noisy network measurements into clean training data suitable for foundation model consumption. The outlier filter implements adaptive exponential moving average baseline tracking with z-score rejection, protecting against both individual measurement anomalies and gradual baseline drift during sparse NTP sampling intervals. Quality thresholds filter measurements with excessive uncertainty (greater than 10 milliseconds default threshold) to maintain prediction model accuracy requirements.

\textbf{Internal Defense Mechanisms:} Foundation models, while powerful, can occasionally produce catastrophic predictions that diverge dramatically from physical clock behavior. ChronoTick implements multi-layer defensive validation constraining predictions to physically plausible ranges while preserving model expressiveness for normal operation. Predictions undergo validation through absolute magnitude thresholds derived from recent measurement history, relative deviation limits compared to observed patterns, and statistical consistency checks against measurement distributions. Predictions exceeding defensive bounds are capped to maximum plausible deviations with appropriately increased uncertainty bounds, preventing individual erroneous forecasts from contaminating the correction system while signaling degraded prediction quality through uncertainty quantification.

This layered defensive approach proved essential in production deployments where network anomalies (congestion, asymmetric routing) create measurement outliers, and foundation models occasionally produce predictions diverging from physical clock behavior. The defenses operate transparently without requiring manual tuning, automatically adapting thresholds based on recent measurement history.

\subsection{Model Independence and Portability}

As foundation model research advances rapidly, ChronoTick's architecture must not be coupled to specific model implementations. Today's state-of-the-art models (TimesFM, Chronos, Moirai, TimeGPT) may be superseded by more capable alternatives, and different deployment environments may favor different model families based on hardware constraints, accuracy requirements, or inference latency budgets. ChronoTick implements a model-agnostic architecture that decouples the synchronization framework from specific foundation model implementations, enabling a ``bring your own model'' deployment strategy.

The core interface requirements impose minimal constraints on model selection: \textit{(i)} time series forecasting that produces multi-step-ahead predictions from historical sequences; \textit{(ii)} uncertainty quantification through probabilistic outputs, confidence intervals, or ensemble variance; and \textit{(iii)} autoregressive prediction that consumes recent historical context to generate forecasts, enabling the retrospective correction mechanism. Beyond these essential requirements, large context windows (512+ historical measurements) improve long-term trend detection, fast inference latency enables tighter prediction scheduling, and zero-shot forecasting eliminates per-deployment training. The current implementation employs TimesFM, selected for its native quantile output, flexible prediction horizons, and production-grade inference performance.


\section{Evaluation}

We evaluate ChronoTick through two complementary phases. First, we benchmark foundation models on clock drift data and validate architectural design choices through component ablation studies, systematically disabling individual features to quantify their contributions to prediction accuracy. This internal validation establishes the optimal system configuration: dual-model architecture with retrospective correction achieving 0.604 ms mean absolute error (MAE), representing 2.7$\times$ improvement over single-model alternatives.

Second, we demonstrate production performance across real-world scenarios: microsecond-scale client access latency validates deployment viability, defensive mechanisms maintain accuracy under measurement anomalies (outlier detection improving MAE by 74\%), unsynchronized clock experiments demonstrate bounded predictions even with natural drift rates reaching -7.877 ms/hour, and sustained 10+ hour deployments achieve 94.9\% uncertainty bound coverage, meeting TrueTime-equivalent probabilistic guarantees while maintaining 2.5-3$\times$ better temporal consistency than commodity NTP implementations.

Experiments span three diverse platforms demonstrating broad applicability: consumer workstations running Windows, cloud servers running Debian, and HPC cluster nodes running Ubuntu. Platform diversity validates zero-shot generalization capability across varying hardware configurations, operating systems, and network characteristics without requiring per-device training or calibration.

\subsection{Experimental Setup}

\textbf{Software.} All evaluations employ TimesFM foundation models (v2.5-200m) for temporal prediction. The system uses Python 3.9-3.12 across platforms with pytorch 2.1.0, timesfm 1.0.0, and numpy 1.24.0. NTP measurements are obtained using ntplib 0.4.0 querying pool.ntp.org servers. The production configuration implements dual-model architecture with inverse variance fusion and retrospective correction via backtracking. Validation methodology makes use of similar external NTP servers for ChronoTick and Chrony (pool.ntp.org), and relies on a separate set of servers (google.ntp.org) to acquire ground truth measurements every minute for comparison. Runs span 1-8 hours collecting thousands of samples. For synchronized platform experiments, we compare against chrony 4.0-4.5 as the baseline NTP implementation.

\textbf{Hardware.} Experiments utilize three distinct platforms representing diverse deployment scenarios:

\textbf{Workstation:} Consumer workstation running Windows with AMD Ryzen 5 3600, 16GB DDR4 RAM, and AMD Radeon 6950XT GPU.

\textbf{Cloud Media Server:} Dedicated server running Debian with Intel Core i7-6700 and 16GB DDR4 RAM.

\textbf{HPC Cluster:} Compute nodes running Ubuntu with dual Intel Xeon Silver 4114 processors and 46GB DDR4 RAM per node. Network connectivity through centralized proxy infrastructure.

\subsection{Foundation Model Benchmarking}

\begin{figure*}[h!]
    \centering
    \begin{subfigure}[b]{0.22\textwidth}
        \centering
        \includegraphics[width=\textwidth, trim=0 0 120 12, clip]{figures/MAE/unsy_cpu_short.png}
        \caption{Short Window CPU}
        \label{subfig:cpu_short}
    \end{subfigure}%
    \begin{subfigure}[b]{0.22\textwidth}
        \centering
        \includegraphics[width=\textwidth, trim=0 0 120 13, clip]{figures/MAE/unsy_cpu_long.png}
        \caption{Long Window CPU}
        \label{subfig:cpu_long}
    \end{subfigure}%
    \begin{subfigure}[b]{0.22\textwidth}
        \centering
        \includegraphics[width=\textwidth, trim=0 0 120 23, clip]{figures/MAE/unsy_gpu_short.png}
        \caption{Short Window GPU}
        \label{subfig:gpu_short}
    \end{subfigure}%
    \begin{subfigure}[b]{0.30\textwidth}
        \centering
        \includegraphics[width=\textwidth, trim=0 0 10 24, clip]{figures/MAE/unsy_gpu_long.png}
        \caption{Long Window GPU}
        \label{subfig:gpu_long}
    \end{subfigure}
    \caption{Foundation model MAE comparison for clock drift prediction across hardware platforms and prediction horizons. Data collected on Chameleon Cloud testbed over 24 hours.}
    \label{fig:mae-comparison}
\end{figure*}

To validate foundation model feasibility for drift prediction, we benchmarked selected models on 24-hour data collected from the Chameleon Cloud testbed~\cite{chamelon_citation}. Figure~\ref{fig:mae-comparison} compares models across short-window (5s, 10s) and long-window (20s, 40s, 60s) horizons on both CPU and GPU. For each model, we compute and average the median absolute error and execution time across 25 randomly selected but consistent samples, evaluating eight context window lengths ranging from 10s to 5 minutes.

In short-horizon analysis, the Chronos family consistently achieves the lowest MAEs. Chronos-mini delivers the best combination of accuracy (MAE of $1.22\times10^{-5}$) and speed (0.0168\,s on GPU; 0.0705\,s on CPU). Moirai achieves comparable error ($2.72\times10^{-5}$) but suffers from high CPU latency (0.824\,s). Time-MoE offers a balanced trade-off with MAE of $1.29\times10^{-5}$ and sub-second inference. TimesFM exhibits higher error ($6.39\times10^{-4}$) but provides native quantile outputs critical for uncertainty quantification. MOMENT shows high error but delivers ultra-fast inference (0.033\,s GPU), suitable for speed-prioritized applications. TTM lacks GPU support and records the highest MAE ($2.37\times10^{-3}$).

In long-horizon analysis, Chronos-Base achieves the lowest MAE of $1.55\times10^{-5}$, followed by Chronos-Small ($1.92\times10^{-5}$) and Chronos-Mini ($2.14\times10^{-5}$). TimesFM's native quantile capability and production-grade performance motivated its selection as ChronoTick's primary engine despite not achieving the lowest raw MAE---the probabilistic outputs are essential for the uncertainty quantification that Graham's polynomial approach~\cite{graham_nsdi22} lacks.

\subsection{Component Ablation Study}

\begin{figure*}[!t]
    \centering
    \includegraphics[width=\textwidth]{figures/2_mae_grouped.pdf}
    \caption{Comprehensive ablation study comparing alternative architectural choices across four design dimensions: model architecture, retrospective correction algorithms, drift rate estimation methods, and baseline smoothing.}
    \label{fig:design_comparison}
\end{figure*}

Before comparing ChronoTick against external baselines, we validate our architectural design choices through systematic ablation. We systematically disable or replace individual components to quantify their contributions to prediction accuracy, establishing the optimal configuration for production deployment. Each configuration runs for 1 hour on the Workstation platform, with mean absolute error between predicted and true clock offsets serving as the primary accuracy metric.

Figure~\ref{fig:design_comparison} compares alternative architectural choices across four design dimensions: model architecture (single minutes-horizon model, single hours-horizon model, or dual-model fusion), retrospective correction algorithms (none, linear interpolation, proportional adjustment, or backtracking), drift rate estimation methods (consecutive measurement differencing versus windowed linear regression over recent predictions), and baseline smoothing (exponential moving average filtering versus no smoothing).

The production configuration (dual-model architecture with backtracking retrospective correction, windowed drift rate estimation, and baseline smoothing) achieves 0.604 ms MAE, establishing this as the optimal configuration. Single-model architectures demonstrate fundamental limitations: the minutes-horizon model alone reaches 1.625 ms MAE (2.7$\times$ worse than production), while the hours-horizon model alone achieves 1.030 ms MAE (1.7$\times$ worse), validating the dual-model fusion design for combining immediate responsiveness with long-range forecasting capability.

Among dual-model variants, retrospective correction algorithms show substantial impact. Backtracking correction (0.604 ms) outperforms proportional adjustment (1.048 ms, 1.7$\times$ worse), linear interpolation (1.271 ms, 2.1$\times$ worse), and no correction (1.296 ms, 2.1$\times$ worse). The windowed drift rate estimation method reduces error from 1.117 ms to 0.604 ms (1.8$\times$ improvement). Baseline smoothing through exponential moving average filtering provides additional refinement, reducing error from 1.157 ms to 0.604 ms (1.9$\times$ improvement).

\subsection{Client Access Performance}

\begin{figure}[!htbp]
    \centering
    \includegraphics[width=0.95\linewidth, trim=0 0 0 23, clip]{figures/1_access_performance.pdf}
    \caption{Performance comparison of ChronoTick's lock-free shared memory IPC architecture against direct NTP queries and native system clock access.}
    \label{fig:access_performance}
\end{figure}

The lock-free IPC shared memory architecture enables microsecond-level client access latency, validating practical deployment viability for latency-sensitive applications. As shown in Figure~\ref{fig:access_performance}, single-client access completes in 2.00 $\mu$s (orders of magnitude faster than direct NTP queries at 42.93 ms) while approaching native system clock access (0.11 $\mu$s). Concurrent access scales linearly from 2.00 $\mu$s (1 client) to 5.50 $\mu$s (8 clients), demonstrating the architecture supports parallel queries without contention or performance degradation.

\subsection{Defensive Mechanisms}


\begin{figure}[t]
    \centering
    \includegraphics[width=0.95\linewidth]{figures/5a_external_defense.pdf}
    \caption{Evaluation of ChronoTick's external defensive mechanisms protecting against corrupted NTP measurements in the HPC cluster environment exhibiting naturally occurring outliers from proxy architecture.}
    \label{fig:external_defense}
\end{figure}

\begin{figure}[t]
    \centering
    \includegraphics[width=0.95\linewidth]{figures/5b_internal_defense.pdf}
    \caption{Evaluation of ChronoTick's internal defensive mechanisms under adversarial system clock chaos induced by conflicting dual synchronizers running simultaneously.}
    \label{fig:internal_defense}
\end{figure}

ChronoTick defends against two threat vectors: external corruption of input measurements and internal unreliability of model predictions. We evaluate external defense within the HPC cluster, which exhibits naturally occurring outliers from its proxy architecture where all nodes access external NTP servers through centralized gateway nodes. ChronoTick employs modified z-score outlier detection (threshold 3.0) to identify and filter anomalous measurements. For internal defense, we evaluate the Workstation under induced system clock chaos from conflicting dual synchronizers (Hyper-V TimeSync and Chrony) running simultaneously for 2 hours, creating erratic training signals that emulate scenarios where model predictions become unreliable.

Figure~\ref{fig:external_defense} shows ChronoTick detecting 34 outliers (14.3\% of samples, 420 ms maximum magnitude). With outlier filtering enabled, ChronoTick achieves 1.030 ms MAE and 1.449 ms RMSE. Without filtering (right, hypothetical), performance degrades to 3.903 ms MAE and 27.671 ms RMSE; a 74\% MAE improvement from defensive filtering.

Figure~\ref{fig:internal_defense} demonstrates Workstation resilience under adversarial conditions. During the 2-hour adversarial period (red shaded), system clock chaos exhibits 332 ms mean error and 231 ms standard deviation (ranging 50-820 ms). ChronoTick maintains 221 ms MAE despite erratic training signals from the conflicting synchronizers. After disabling the adversarial system (purple dashed line), the system requires approximately 2 hours to fully restore accuracy (green shaded recovery period) as retrospective correction gradually rebuilds clean historical context from valid NTP measurements, demonstrating the mechanism's self-healing capability.

These results validate that multi-layer validation successfully filters anomalous measurements while uncertainty quantification and retrospective correction enable adaptation when measurement quality improves or degrades, transforming ChronoTick into a production-hardened service maintaining temporal consistency despite hostile conditions.

\subsection{Holdover Performance}


\begin{figure}[t]
    \centering
    \includegraphics[width=0.95\linewidth, trim=0 0 0 23, clip]{figures/exp13_homelab_unsync.pdf}
    \caption{Validation of ChronoTick's predictive capability on unsynchronized Cloud Media Server exhibiting natural clock drift by transitioning from 8-hour unsynchronized operation to synchronized deployment with system NTP disabled.}
    \label{fig:unsync_test}
\end{figure}

Before evaluating sustained synchronized deployments, we validate ChronoTick's predictive capability on a clock exhibiting natural drift by transitioning from unsynchronized to synchronized operation. This experiment demonstrates the system's ability to track clock behavior and provide bounded error even without valid external synchronization---a fundamental requirement for reducing synchronization frequency in resource-constrained environments.

Figure~\ref{fig:unsync_test} presents an 8-hour unsynchronized deployment on the Cloud Media Server with system NTP disabled. The system clock diverges at -7.877 ms/hour (a natural drift rate typical of commodity hardware), and ChronoTick maintains 16.022 ms MAE throughout the experiment, demonstrating that foundation models can extract sufficient temporal structure for bounded predictions without requiring well-behaved synchronized clocks---a critical holdover scenario that validates drift prediction on free-running oscillators.


\subsection{Sustained Production Deployments}


\begin{figure*}[!t]
    \centering
    \begin{subfigure}[b]{\textwidth}
        \centering
        \includegraphics[width=0.95\textwidth]{figures/longterm_node1_offset.pdf}
        \caption{Node 1 offset behavior (ares-comp11-fix1)}
        \label{subfig:longterm_node1_offset}
    \end{subfigure}

    \vspace{0.5em}

    \begin{subfigure}[b]{0.48\textwidth}
        \centering
        \includegraphics[width=\textwidth]{figures/longterm_node1_cumulative.pdf}
        \caption{Node 1 cumulative error}
        \label{subfig:longterm_node1_cum}
    \end{subfigure}%
    \hfill
    \begin{subfigure}[b]{0.48\textwidth}
        \centering
        \includegraphics[width=\textwidth]{figures/coverage_chronotick.png}
        \caption{Node 1 coverage analysis}
        \label{subfig:longterm_node1_coverage}
    \end{subfigure}
    \caption{Sustained 10-hour production deployment results on HPC cluster Node 1 demonstrating end-to-end ChronoTick capability against chrony.}
    \label{fig:longterm_node1}
\end{figure*}

\begin{figure*}[!t]
    \centering
    \begin{subfigure}[b]{\textwidth}
        \centering
        \includegraphics[width=0.95\textwidth]{figures/longterm_node2_offset.pdf}
        \caption{Node 2 offset behavior (ares-comp12-fix1)}
        \label{subfig:longterm_node2_offset}
    \end{subfigure}

    \vspace{0.5em}

    \begin{subfigure}[b]{0.48\textwidth}
        \centering
        \includegraphics[width=\textwidth]{figures/longterm_node2_cumulative.pdf}
        \caption{Node 2 cumulative error}
        \label{subfig:longterm_node2_cum}
    \end{subfigure}%
    \hfill
    \begin{subfigure}[b]{0.48\textwidth}
        \centering
        \includegraphics[width=\textwidth]{figures/longterm_node1_uncertainty_clean.pdf}
        \caption{Node 2 coverage analysis}
        \label{subfig:longterm_node2_coverage}
    \end{subfigure}
    \caption{Sustained 10-hour production deployment results on HPC cluster Node 2 demonstrating end-to-end ChronoTick capability against chrony.}
    \label{fig:longterm_node2}
\end{figure*}

Having validated incremental design decisions, defensive mechanisms, and unsynchronized performance, we now evaluate end-to-end production capability through sustained 10+ hour deployments on synchronized systems. These experiments demonstrate ChronoTick's ability to supersample clock behavior: using sparse NTP measurements combined with foundation model predictions to achieve sub-millisecond precision between synchronization points.

Figures~\ref{fig:longterm_node1} and~\ref{fig:longterm_node2} present results from two HPC cluster compute nodes over 10-hour continuous deployments. Both nodes maintain synchronized clocks via chrony while ChronoTick provides predictive corrections between NTP measurements (arriving every minute). Node 1 achieves 0.872 ms MAE with 1.568 ms RMSE, while Node 2 achieves 0.604 ms MAE with 0.977 ms RMSE, demonstrating sub-millisecond precision through the dual-model architecture combining minutes-horizon and hours-horizon forecasting.

The cumulative error plots (subfigures b) show bounded error accumulation throughout 10-hour deployments, demonstrating that ChronoTick maintains temporal consistency 2.5-3$\times$ better than commodity NTP implementations. The coverage analysis (subfigures c) validates well-calibrated uncertainty bounds that accurately reflect prediction confidence: the system successfully bounds 94.9\% of all predictions within stated confidence intervals through per-prediction uncertainty quantification. This 94.9\% coverage meets TrueTime-equivalent probabilistic guarantees, enabling distributed algorithms to reason about temporal uncertainty.

These sustained deployments validate the complete system narrative: foundation models trained on diverse temporal data successfully generalize to clock synchronization across heterogeneous platforms without per-device training, achieving bounded sub-millisecond precision with sparse NTP measurements while providing calibrated uncertainty estimates---addressing all seven limitations of polynomial-only approaches.

\section{Conclusions}

We presented ChronoTick, a system that applies time series foundation models to multivariate oscillator drift prediction on commodity hardware. ChronoTick addresses all seven limitations of prior polynomial-based compensation:

\begin{enumerate}
    \item \textbf{Beyond polynomials}: A dual-model architecture with inverse variance fusion captures complex nonlinear dynamics including transient variations and long-term trends that fixed parametric models cannot represent.
    \item \textbf{Zero per-device calibration}: Zero-shot foundation model inference eliminates the 48-hour calibration requirement, enabling immediate deployment across heterogeneous hardware.
    \item \textbf{Hysteresis and transient capture}: The short-horizon model with frequent updates captures rapid thermal transients and hysteresis effects through multivariate context.
    \item \textbf{Online aging adaptation}: Retrospective correction continuously refines the prediction dataset, adapting to crystal aging without manual recalibration.
    \item \textbf{Multivariate sensors}: ChronoTick leverages CPU temperature, frequency, load, and multiple thermal zones---not just a single DIMM sensor.
    \item \textbf{Fully open source}: The complete pipeline from data collection through model inference to real-time correction is publicly available.
    \item \textbf{Uncertainty-aware}: Quantile-based uncertainty extraction provides calibrated confidence intervals achieving 94.9\% coverage, enabling downstream algorithms to reason about correction quality.
\end{enumerate}

Evaluation across three heterogeneous platforms (consumer workstation, cloud server, HPC cluster) demonstrates 0.604\,ms MAE without per-device training, representing 2.5--3$\times$ improvement over commodity NTP. Defensive mechanisms maintain accuracy under measurement anomalies (74\% MAE improvement from outlier filtering), and sustained 10+ hour deployments validate production-ready operation with TrueTime-equivalent probabilistic guarantees.

These results demonstrate that modern ML---specifically zero-shot time series foundation models---can replace manual calibration and static polynomial fitting for software-defined oscillator compensation, democratizing precision timing for commodity servers without specialized hardware investments.


\bibliographystyle{IEEEtran}
\bibliography{chronotick}

\end{document}
