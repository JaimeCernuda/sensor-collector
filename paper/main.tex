\documentclass[conference]{IEEEtran}
\IEEEoverridecommandlockouts

\usepackage{graphicx}
\usepackage{booktabs}
\usepackage{amsmath}
\usepackage{algorithm}
\usepackage{algpseudocode}
\usepackage{subcaption}
\usepackage{multirow}
\usepackage{hyperref}
\usepackage{xcolor}
\usepackage{colortbl}
\usepackage{pifont}

\hypersetup{
    colorlinks=true,
    linkcolor=blue,
    urlcolor=blue,
    citecolor=blue
}

% Table markers
\newcommand{\cmark}{\cellcolor{green!20}{\ding{51}}}
\newcommand{\xmark}{\cellcolor{red!20}{\ding{55}}}
\newcommand{\nmark}{\cellcolor{yellow!30}N/A}

\begin{document}

\title{ChronoTick: Software-Defined Oscillator Compensation via Foundation Model Drift Prediction}

\author{
\IEEEauthorblockN{Jaime Cernuda}
\IEEEauthorblockA{\textit{Department of Computer Science} \\
\textit{Illinois Institute of Technology}\\
Chicago, IL, USA \\
jcernudagarcia@illinoistech.edu}
\and
\IEEEauthorblockN{Anthony Kougkas}
\IEEEauthorblockA{\textit{Department of Computer Science} \\
\textit{Illinois Institute of Technology}\\
Chicago, IL, USA \\
akougkas@illinoistech.edu}
\and
\IEEEauthorblockN{Xian-He Sun}
\IEEEauthorblockA{\textit{Department of Computer Science} \\
\textit{Illinois Institute of Technology}\\
Chicago, IL, USA \\
sun@illinoistech.edu}
}

\maketitle

\begin{abstract}
Commodity servers require accurate clocks for distributed applications such as databases, tracing systems, and observability pipelines, yet most deployments lack specialized timing infrastructure such as PTP-capable switches or GPS receivers. Network Time Protocol periodically corrects clock offset, but oscillator drift accumulates between synchronization events. Existing software compensation approaches require per-device calibration, rely on single sensors, and provide no uncertainty quantification. We present ChronoTick, a system that applies zero-shot time series foundation models to multivariate oscillator drift prediction on commodity hardware. Pre-trained models eliminate per-device calibration, enabling immediate deployment. Retrospective correction refines predictions as ground truth from synchronization events becomes available, enabling continuous adaptation to changing thermal and workload conditions. Integration of diverse hardware monitoring streams (temperatures, CPU frequency, memory pressure, idle states, I/O) captures joint dynamics that single-sensor approaches miss. Across heterogeneous platforms, ChronoTick achieves sub-millisecond accuracy (0.49--0.91\,ms mean absolute error), 2.4--3.3$\times$ improvement over commodity NTP, and 94.9\% uncertainty coverage. ChronoTick is released as open source.
\end{abstract}

\begin{IEEEkeywords}
oscillator drift prediction, clock compensation, time series foundation models, NTP, bounded uncertainty
\end{IEEEkeywords}

\section{Introduction}

The growing demands of distributed applications have driven a need for tightly synchronized clocks. Instead of coordinating over the network, applications can establish event ordering using timestamps from local clocks, but only if those clocks are sufficiently accurate. From distributed databases~\cite{cockroachdb_hlc_implementation} and tracing systems~\cite{dapper_google2010} to log stores~\cite{chronolog_msst2020, xplog_tsc2026}, scientific computing workflows~\cite{denis_starpu_tracing2023}, or agentic AI system observability, where the community is moving towards chains of thousands of asynchronous autonomous events across process boundaries~\cite{agentsight_pacmi2025}. When clocks disagree, events appear out of order, causal chains break, and system behavior becomes difficult to diagnose or reproduce. Achieving sub-millisecond accuracy has traditionally required specialized infrastructure such as PTP-capable network interfaces, GPS receivers, or atomic references~\cite{ptp}, yet commodity servers, edge deployments, and heterogeneous clusters lack access to such hardware.

Software-based compensation of the local oscillator offers an alternative path: correcting drift using sensors already present on commodity motherboards. Between synchronization events, commodity servers free-run on local quartz oscillators whose frequency varies with temperature, supply voltage, and aging~\cite{tirado_andres_clock_sources_2019}. NTP~\cite{ntp} periodically corrects clock offset over the network, but drift accumulates during holdover. Hardware solutions such as PTP-synchronized networks, GPS-disciplined references, and atomic clocks can achieve nanosecond-level accuracy; datacenter systems such as Sundial~\cite{sundial_osdi2020} and Huygens~\cite{huygens_nsdi2018_geng_svm_nanosec} demonstrate what becomes possible with dedicated timing infrastructure. For the vast majority of commodity deployments where such infrastructure is impractical, software compensation has emerged as a practical alternative. Chrony's \texttt{tempcomp} directive~\cite{ntp} applies a static polynomial to a single temperature reading. Graham~\cite{graham_nsdi22} reported that automatic cubic polynomial fitting to DIMM temperature readings could reduce commodity oscillator drift by up to three orders of magnitude. Across adjacent domains, from satellite clock bias prediction~\cite{he_bds3_lstm_72pct_arima, gnss_clock_prediction_review_2025} to TCXO frequency compensation~\cite{su_lstm_tcxo_transfer_learning}, machine learning has consistently improved over polynomial baselines.

Yet existing software compensation approaches share limitations that constrain practical deployment. Per-device calibration remains a barrier, with current systems requiring extended learning periods per server before compensation can begin~\cite{graham_nsdi22, su_lstm_tcxo_transfer_learning}. Current methods are designed for single-variable correction, typically relying on a single temperature sensor without addressing the joint dynamics of other node variables such as CPU frequency, memory pressure, idle states, and I/O load~\cite{tirado_andres_clock_sources_2019}. Once the compensation model is fit, it remains fixed; thermal conditions, workload patterns, and environmental factors shift during operation~\cite{tirado_andres_clock_sources_2019}, yet no existing approach provides mechanisms for continuous self-improvement or retrospective correction. Finally, state-of-the-art solutions are either closed-source~\cite{graham_nsdi22} or tied to proprietary infrastructure~\cite{huygens_nsdi2018_geng_svm_nanosec, sundial_osdi2020}; there is a clear need for an accessible precision software clock on commodity hardware~\cite{najafi_hotos2021_time}.

Time series foundation models offer a new capability for addressing these gaps. Pre-trained on large and diverse corpora, models such as Chronos~\cite{Chronos}, TimesFM~\cite{TimesFM}, MOIRAI~\cite{MOIRAI}, and MOMENT~\cite{MOMENT} learn generalizable temporal patterns that transfer to unseen domains without per-device training or fine-tuning. Models such as TimesFM and MOIRAI natively support multivariate input, jointly modeling cross-variable dynamics, while others such as Chronos operate on individual channels. PatchTST~\cite{PatchTST} achieves strong transfer performance through patch-based architecture, and Time-MoE~\cite{Time-MoE} demonstrates billion-parameter scale, illustrating the breadth and maturity of the field. These models provide probabilistic prediction intervals alongside point forecasts. Rather than static polynomial fitting, they operate through continuous inference on streaming data, enabling a shift from per-device calibration to adaptive learned prediction that generalizes across hardware. Open pre-trained weights make such systems reproducible and accessible. To the best of our knowledge, no prior work has applied time series foundation models to oscillator drift prediction on commodity hardware.

We present ChronoTick, a system that applies zero-shot time series foundation models to predict and correct oscillator drift on commodity hardware, requiring no per-device calibration and no specialized infrastructure. ChronoTick runs multiple foundation models in parallel, fusing their predictions and refining estimates retrospectively as ground truth from synchronization events becomes available. It consumes multivariate input from Linux hwmon sensor streams, including thermal zones, CPU frequency, and system load, and integrates directly with chrony for real-time NTP correction. We release ChronoTick as open source. Across heterogeneous platforms and without per-device training, ChronoTick achieves sub-millisecond accuracy (0.49--0.91\,ms mean absolute error) and 2.4--3.3$\times$ better offset stability than commodity NTP.

This paper makes the following contributions:
\begin{enumerate}
\item \textbf{Zero-shot foundation model drift prediction.} To our knowledge, the first application of pre-trained time series foundation models to commodity oscillator compensation, eliminating per-device calibration entirely.
\item \textbf{Self-improving prediction pipeline.} A pipeline that retrospectively refines drift estimates as ground truth from synchronization events becomes available, enabling continuous adaptation to changing conditions without manual recalibration.
\item \textbf{Multivariate compensation from commodity sensors.} Integration of multiple hardware monitoring streams to capture the joint thermal, frequency, and workload dynamics that single-sensor approaches miss.
\end{enumerate}

\section{Background and Related Work}

\subsection{Clock Synchronization Fundamentals}

Commodity servers rely on quartz crystal oscillators whose frequency drifts with temperature, supply voltage, and aging~\cite{tirado_andres_clock_sources_2019}. The Network Time Protocol (NTP)~\cite{ntp} compensates for this drift through a hierarchical stratum architecture in which clients periodically exchange timestamps with upstream servers. Each exchange yields four timestamps, from which the client estimates its offset as $o = ((t_2 - t_1) + (t_3 - t_4)) / 2$, where $t_1$ and $t_4$ are recorded by the local clock and $t_2$ and $t_3$ by the server. Modern implementations such as chrony achieve 1--10\,ms accuracy on local-area networks through statistical filtering of successive exchanges~\cite{ntp_performance_chrony_ntpd}. However, software timestamping introduces jitter from operating system scheduling and network stack traversal, and asymmetric network paths produce systematic errors that the four-timestamp model cannot distinguish from clock offset. These are design tradeoffs rather than failures: NTP prioritizes ubiquitous deployment on general-purpose networks over maximum precision.

The Precision Time Protocol (PTP, IEEE~1588)~\cite{ptp} closes this accuracy gap by timestamping packets at the MAC layer before operating system involvement, eliminating software-induced jitter. Realizing PTP's sub-microsecond accuracy requires substantial infrastructure investment: PTP-capable network interface cards, switches with transparent clock support, GPS-disciplined grandmaster clocks, and strict network topology control to maintain the symmetric paths PTP assumes~\cite{ptp_hardware_costs_survey, meta_ptp_time_appliances}. Extensions from White Rabbit~\cite{whiterabbit} to Google's TrueTime~\cite{spanner_osdi2012_truetime_gps_atomic} push accuracy further with specialized hardware. These solutions deliver excellent accuracy but confine high-precision timing to controlled environments. Between NTP's millisecond accuracy and PTP's sub-microsecond capability lies a three-order-of-magnitude precision gap that affects edge deployments, heterogeneous clusters, and any environment where dedicated timing hardware is impractical. Software-defined compensation of the local oscillator offers a path to bridge this gap using sensors already present on commodity hardware.

\subsection{Software Clock Compensation}

The temperature-frequency relationship of quartz oscillators has been exploited for compensation for decades. Chrony's \texttt{tempcomp} directive applies a quadratic polynomial to a single \texttt{hwmon} sensor reading, yielding a 3.54$\times$ improvement in offset standard deviation~\cite{ntp_performance_chrony_ntpd}. Commodity motherboards expose thermal sensors through the Linux \texttt{hwmon} subsystem, making software-defined compensation practical without additional hardware. However, clock drift is not governed by temperature alone; supply voltage fluctuations, crystal aging, and workload-induced thermal transients all contribute to frequency instability~\cite{tirado_andres_clock_sources_2019}, and a single quadratic polynomial over one sensor can capture only a fraction of this complexity.

In a landmark study, Graham~\cite{graham_nsdi22} demonstrated that automatically fitting cubic polynomials to DIMM temperature readings reduces commodity oscillator drift by up to three orders of magnitude, from 200\,ppm to approximately 100\,ppb. A key finding was that DIMM temperature sensors serve as better proxies for crystal temperature than CPU die sensors, because the memory modules sit closer to the oscillator on the motherboard and exhibit lower thermal lag. Each device requires individual calibration over approximately 48 hours because manufacturing variance in AT-cut angle produces unique temperature-frequency curves. Once fitted, Graham's polynomial remains fixed; it relies on a single sensor and does not adapt to changing thermal conditions, aging, or workload shifts. The system provides no uncertainty bounds on its predictions and was not released as open source, limiting opportunities for community extension and adaptation to new hardware platforms. Najafi et al. later argued that the systems community urgently needs better open-source timing tools~\cite{najafi_hotos2021_time}, a call that remains largely unanswered.

Beyond software compensation, alternative approaches trade different resources for different guarantees. Google's TrueTime provides bounded clock uncertainty through GPS receivers and atomic clocks~\cite{spanner_osdi2012_truetime_gps_atomic}, achieving nanosecond-level precision at the cost of specialized hardware. Systems such as CockroachDB~\cite{cockroachdb_hlc_implementation} and YugabyteDB~\cite{yugabytedb_hlc_distributed} avoid the hardware cost entirely by using Hybrid Logical Clocks for causal consistency, but sacrifice real-time semantics.

\subsection{Machine Learning for Timing Prediction}

Across multiple timing domains, learned models have achieved substantial gains over analytical baselines. In satellite clock prediction, LSTM and attention-augmented architectures have achieved 72-84\% improvement over polynomial baselines~\cite{he_bds3_lstm_72pct_arima, wang_lstm_attention_bds_2024, li_multivariate_cnn_lstm_gps_2024}, a trend confirmed by recent surveys~\cite{gnss_clock_prediction_review_2025}. In the oscillator domain, Su et al.~\cite{su_lstm_tcxo_transfer_learning} applied LSTM with transfer learning to TCXO compensation, reducing calibration requirements while improving accuracy by 51\%. At the hardware level, Jung et al.~\cite{jung_ml_rco_isscc_2022} achieved 0.68\,ppm/$^\circ$C stability through ML-based RC oscillator calibration, demonstrating that learned compensation has begun crossing from software into silicon.

Within datacenter timing, Huygens~\cite{huygens_nsdi2018_geng_svm_nanosec} applied support vector machines to filter network jitter, achieving tens-of-nanoseconds accuracy, far exceeding what software compensation alone can deliver. However, Huygens operates reactively on current synchronization error rather than predicting future drift, and requires PTP-capable hardware timestamps as input, positioning it as a complementary technique for environments with timing infrastructure rather than a commodity-hardware solution. Notably, every approach in this trajectory trains task-specific models from scratch on domain-specific data; none leverages pre-trained foundation models that generalize across domains without per-task training.

\subsection{Time Series Foundation Models}

Time series foundation models (TSFMs) represent a shift from per-task training to zero-shot generalization on unseen domains. Early work repurposed large language models for temporal prediction~\cite{TIME-LLM, TEMPO, AutoTimes}, and TimeGPT-1~\cite{TimeGPT-1} demonstrated that purpose-built foundation models achieve accurate zero-shot forecasts without domain-specific training. Pre-trained on large, diverse corpora, modern TSFMs address the three key dimensions identified in software clock compensation. First, zero-shot prediction eliminates per-device calibration: Chronos~\cite{Chronos} and MOMENT~\cite{MOMENT} achieve this through tokenization and masked modeling respectively, while TimesFM~\cite{TimesFM}, pretrained on 100 billion real-world time points, extends zero-shot capability to long-horizon forecasting. Second, multivariate input handling addresses the single-sensor limitation: MOIRAI~\cite{MOIRAI} processes arbitrary numbers of input channels through any-variate attention, and several architectures explicitly model exogenous covariates and cross-dimensional dependencies~\cite{Crossformer, TTMs, TimeXer, PatchTST}. Third, probabilistic forecasts provide the uncertainty bounds that polynomial approaches cannot offer: TimesFM produces native quantile outputs, and Time-MoE~\cite{Time-MoE} maintains both point and probabilistic prediction quality at billion-parameter scale.

To our knowledge, no prior work has applied pre-trained, zero-shot time series foundation models to commodity server oscillator drift prediction. ChronoTick bridges this gap.


\section{ChronoTick Design}

ChronoTick is a predictive time synchronization system that applies time series foundation models to forecast oscillator drift before it manifests. Where reactive approaches correct drift after detection, relying on periodic synchronization events to measure and compensate accumulated error, ChronoTick anticipates clock behavior by learning temporal patterns from streaming sensor data. The system operates as a daemon that publishes corrected timestamps through a lock-free shared memory interface, integrating transparently with existing applications at sub-microsecond read latency.

The key insight is increasing effective synchronization frequency: typical NTP deployments poll every 1 to 10 minutes~\cite{ntp}, leaving the local oscillator to free-run between corrections, yet drift accumulates continuously. ChronoTick fills these gaps by generating predictions every second, analogous to how frame interpolation increases effective frame rate from sparse keyframes. By predicting clock behavior between NTP measurements rather than waiting for the next synchronization event, the system achieves sub-millisecond precision on commodity hardware with no specialized NICs, GPS receivers, or atomic clocks. Three contributions make this possible and correspond directly to the architectural components that follow: zero-shot foundation model prediction eliminates per-device calibration (Prediction Layer, Section~3.5), retrospective correction enables continuous self-improvement as ground truth arrives (Section~3.7), and multivariate compensation from commodity sensors captures the full environmental context that single-sensor approaches miss (Data Collection, Section~3.3).

\subsection{Architecture}

\begin{figure}[h]
    \centering
    \includegraphics[width=\linewidth]{figures/chronotick_arch.pdf}
    \caption{ChronoTick predictive synchronization architecture.}
    \label{fig:arch_htick}
\end{figure}

The architecture of ChronoTick, shown in Figure~\ref{fig:arch_htick}, implements a five-layer predictive synchronization stack: the \textit{Client Interface}, \textit{Scheduler}, \textit{Prediction Layer}, \textit{Retrospective Correction}, and \textit{Data Collection}. This design prioritizes response latency through predictive scheduling that decouples model inference latency from client request response times, enabling the system to serve as a viable replacement for operating system time calls across diverse deployment scenarios.

The \textbf{Client Interface} addresses the scalability challenge that operating system time calls complete in nanoseconds yet ChronoTick must serve arbitrarily many concurrent clients at comparable latencies. A daemon architecture with lock-free shared memory enables concurrent reads of pre-computed corrections without mutual exclusion overhead.

The \textbf{Scheduler} resolves the orders-of-magnitude mismatch between model inference timescales (tens of milliseconds) and client query latencies (sub-millisecond) by pre-computing forecasts before clients request them. Predictive scheduling eliminates inference latency from the critical path.

At the heart of the system, the \textbf{Prediction Layer} employs a dual-model architecture balancing temporal responsiveness with prediction stability. Short-horizon and long-horizon models leverage time series foundation models for zero-shot prediction, producing point estimates and uncertainty bounds through probabilistic quantile outputs. Predictions combine via inverse variance weighted fusion, automatically adapting to model confidence.

The \textbf{Retrospective Correction} maintains dataset quality by validating predictions against NTP measurements, then correcting the historical dataset by interpolating between ground truth anchor points. This corrected dataset becomes the context for subsequent predictions, enabling autoregressive models to learn from mistakes and adapt to changing system dynamics.

The \textbf{Data Collection} harvests timing data from network protocols and system performance metrics with dedicated CPU affinity to minimize jitter. Multivariate sensors (temperatures, CPU frequency, memory pressure, idle states, I/O) enable models to distinguish predictable environmental effects from genuine timing anomalies, a key differentiator versus Graham's single sensor approach~\cite{graham_nsdi22}.

\subsection{Client Interface}

ChronoTick implements a daemon architecture that pre-computes corrections in a dedicated process with CPU affinity, isolating prediction computation from client workloads. The daemon operates continuously, surviving client failures while serving all active clients. This separation enables the system to replace operating system time calls without coupling inference latency to response latency.

Lock-free shared memory enables concurrent reads without mutual exclusion overhead. The system implements a single-writer-multiple-reader pattern: the daemon writes pre-computed corrections to cache-aligned shared memory while arbitrarily many clients read concurrently without locks, atomic operations, or coordination overhead. Sequence number validation detects torn reads during concurrent updates, enabling readers to retry when writes interrupt reads. This design achieves sub-microsecond read latency. The shared memory buffer contains corrections, uncertainty bounds, drift rates, validity periods, and operational status.

\subsection{Multivariate Data Collection}

The data collection layer acquires timing measurements optimized for foundation model input requirements, collecting synchronized timing data alongside system performance metrics with controlled frequency during warmup and operational phases. ChronoTick independently queries NTP servers and predicts the drift that accumulates between its own synchronization measurements. Because NTP polls occur at configurable intervals (every 5 seconds during warmup, less frequently during normal operation), drift between measurements is bounded and exhibits learnable temporal patterns, making it a favorable target for foundation model forecasting. Collection operates on the daemon CPU core with elevated priority to minimize measurement jitter, which proves critical for accurate drift pattern detection by forecasting models.

Unlike prior approaches that rely on a single temperature sensor~\cite{graham_nsdi22}, ChronoTick collects multivariate system telemetry: CPU temperatures across multiple thermal zones, processor frequencies, idle state residency, memory pressure, I/O throughput, and additional hwmon sensors when available. This multivariate approach captures the full environmental context affecting oscillator behavior, including thermal transients, frequency scaling effects on timing circuits, and cross-correlated system dynamics that single-sensor polynomial approaches miss.

ChronoTick implements NTP measurement querying multiple authoritative time servers using standard synchronization protocols. Each measurement exchange computes clock offset using the four-timestamp calculation~\cite{ntp}: $o = ((t_2 - t_1) + (t_3 - t_4)) / 2$, where $t_1$ and $t_4$ represent local timestamps while $t_2$ and $t_3$ represent server timestamps. Network delay and measurement uncertainty follow from $d = (t_4 - t_1) - (t_3 - t_2)$ and $\sigma_{\text{measurement}} = \max(d/2, \sigma_{\text{server}})$ respectively. Multiple servers are queried with outlier rejection using median absolute deviation filtering to eliminate anomalous measurements caused by network congestion. The median offset with mean round-trip delay provides improved measurement quality, reducing uncertainty.

During initial deployment, no historical context exists for model forecasting. The system operates in warmup mode with elevated NTP polling at 5-second intervals, delivering corrections from direct measurements without model predictions. After 60 seconds accumulating 12 measurements, the system transitions to normal operation, activating predictive scheduling and retrospective correction.

\subsection{Scheduler}

Foundation model inference requires tens of milliseconds while applications demand sub-millisecond response latency; ChronoTick resolves this mismatch by pre-computing forecasts before clients need them. The scheduling architecture operates through self-rescheduling prediction tasks: each model schedules its next prediction upon completing current inference. The short-horizon model schedules frequent predictions, minimizing cache staleness. The long-horizon model schedules infrequent predictions further in advance.

The scheduler manages a rolling prediction cache centered on current time, evicting past forecasts while retaining future ones. Cache hits deliver corrections with sub-millisecond latency through temporal lookup, while cache misses trigger on-demand prediction at the cost of increased response latency.

\subsection{Prediction Layer}

ChronoTick implements a dual-model forecasting architecture that applies pre-trained time series foundation models to predict clock drift patterns before they occur, requiring no per-device calibration or fine-tuning. Because these models learn generalizable temporal patterns from large corpora, they transfer directly to unseen oscillator environments in a zero-shot fashion, enabling proactive synchronization on any commodity server from initial deployment. The short-horizon model operates with frequent updates, limited historical context, and system metrics inputs, capturing rapid fluctuations and transient anomalies while maintaining low inference latency for immediate corrections. The long-horizon model updates infrequently with extended prediction windows and substantial historical context, providing smooth and stable drift trend estimation resistant to measurement noise and short-term variations, prioritizing prediction stability over responsiveness. Both models produce probabilistic forecasts through multi-quantile outputs that directly encode prediction uncertainty, enabling direct extraction of confidence intervals without separate uncertainty estimation procedures.

\subsubsection{Prediction Inputs and Outputs}

Before describing how ChronoTick corrects system timestamps, we clarify the inputs and outputs of the prediction layer. The prediction layer consumes two primary time series (historical skew in ms and drift rate in ppm, spanning the model's context window) alongside multivariate sensor covariates. Available sensor categories include CPU core, package, and peripheral temperatures, CPU operating frequencies, processor idle state (C-state) residency, memory pressure, I/O throughput (disk and network), CPU load, and system counters. The exact sensor set varies by platform and privilege level; ChronoTick ingests all available streams through Linux hwmon, procfs, and sysfs interfaces (or platform-specific equivalents on Windows).

\begin{table}[h]
\centering
\footnotesize
\caption{Prediction Layer Outputs}
\label{tab:prediction_outputs}
\begin{tabular}{@{}lp{2.8cm}ll@{}}
\toprule
\textbf{Output} & \textbf{Description} & \textbf{Symbol} & \textbf{Units} \\ \midrule
Predicted skew & Clock offset from true time & $\hat{s}$ & ms \\
Predicted drift & Rate of clock divergence & $\hat{r}$ & ppm \\
Skew uncertainty & Skew confidence bound & $\sigma_s$ & ms \\
Drift uncertainty & Drift confidence bound & $\sigma_r$ & ppm \\
\bottomrule
\end{tabular}
\end{table}

The predicted \textit{skew} $\hat{s}$ represents the instantaneous offset between the system clock and true time at the moment of prediction. The predicted \textit{drift} $\hat{r}$ represents the rate at which the system clock diverges from true time, measured in parts per million (ppm); for example, 10 ppm means the clock gains or loses 10 microseconds per second. These are distinct quantities: skew is corrected by adding an offset to the system clock, while drift is used to compensate for time elapsed since the nearest prediction. The foundation models predict both quantities simultaneously through multi-horizon forecasting, with uncertainty bounds $\sigma_s$ and $\sigma_r$ extracted directly from quantile outputs. Historical sequences span the model's context window, typically 512 measurements at 1\,Hz sampling for the long-horizon model.

\subsubsection{Quantile-Based Uncertainty Extraction}

Foundation models produce quantile predictions at multiple probability levels; ChronoTick uses the median as the point prediction and the quantile spread to estimate uncertainty. The extraction follows:
\begin{equation}
\hat{y} = q_{0.5}
\end{equation}
\begin{equation}
\sigma = \frac{q_{0.9} - q_{0.1}}{2.56}
\end{equation}
The divisor 2.56 arises because the 10th-to-90th percentile span covers $2 \times \Phi^{-1}(0.9) \approx 2.56$ standard deviations under a Gaussian assumption.

\subsubsection{Inverse Variance Fusion}

The system fuses short- and long-horizon predictions using inverse variance weighting, which is provably optimal for combining unbiased estimators:
\begin{align}
w_i &= \frac{1/\sigma_i^2}{\sum_j 1/\sigma_j^2} \\
\hat{y}_{\text{fused}} &= w_{\text{short}} \hat{y}_{\text{short}} + w_{\text{long}} \hat{y}_{\text{long}} \\
\sigma_{\text{fused}} &= \frac{1}{\sqrt{1/\sigma_{\text{short}}^2 + 1/\sigma_{\text{long}}^2}}
\end{align}
The fused uncertainty is always smaller than either individual uncertainty, and when one model reports higher uncertainty the fusion automatically weights the more confident prediction.

\subsection{Time Correction Mechanism}

ChronoTick synthesizes corrected physical time by applying predicted skew and drift compensation to the system clock. The correction engine computes timestamps by applying a two-step process. First, it corrects the current system clock using the nearest predicted skew $\hat{s}_{\text{nearest}}$ from the prediction cache. Second, it compensates for elapsed time since that prediction using the predicted drift rate $\hat{r}_{\text{drift}}$; this ``walking forward'' process accounts for how the system clock continues to diverge during the interval between the prediction time and the query time.

\begin{algorithm}[H]
\caption{System Clock Correction with Predictive Skew}
\label{algo:time_correction}
\begin{algorithmic}[1]
\Require Current system time $t_{\text{system}}$, Nearest predicted skew $\hat{s}_{\text{nearest}}$ at $t_{\text{nearest}}$, Predicted drift rate $\hat{r}_{\text{drift}}$ (ppm), Skew uncertainty $\sigma_s$, Drift uncertainty $\sigma_r$, Last returned time $t_{\text{prev}}$, Minimum increment $\epsilon$
\Ensure Corrected time $t_{\text{corrected}}$, Total uncertainty $\sigma_{\text{total}}$
\State $\Delta t \gets t_{\text{system}} - t_{\text{nearest}}$ \Comment{Elapsed since nearest prediction}
\State $t_{\text{corrected}} \gets t_{\text{system}} + \hat{s}_{\text{nearest}} + \hat{r}_{\text{drift}} \cdot \Delta t$ \Comment{Apply skew + drift compensation}
\State $\sigma_{\text{total}} \gets \sqrt{\sigma_s^2 + (\sigma_r \cdot \Delta t)^2}$ \Comment{Uncertainty propagation}
\If{$t_{\text{corrected}} \leq t_{\text{prev}}$} \Comment{Monotonicity enforcement}
    \State $t_{\text{corrected}} \gets t_{\text{prev}} + \epsilon$
    \State $\sigma_{\text{total}} \gets \sigma_{\text{total}} + |t_{\text{corrected}} - (t_{\text{system}} + \hat{s}_{\text{nearest}} + \hat{r}_{\text{drift}} \cdot \Delta t)|$ \Comment{Inflate uncertainty}
\EndIf
\State \Return $t_{\text{corrected}}, \sigma_{\text{total}}$
\end{algorithmic}
\end{algorithm}

Total uncertainty combines prediction uncertainty with temporal degradation: $\sigma^2_{\text{total}} = \sigma^2_s + \sigma^2_r \cdot (\Delta t)^2$.

Because ChronoTick's own NTP measurements continuously anchor the prediction pipeline, corrections remain small adjustments bounded by the drift accumulated since the last synchronization event. Algorithm~\ref{algo:time_correction} prevents backward time jumps: if a corrected timestamp would not exceed the last returned value $t_{\text{prev}}$, the algorithm clamps the output to $t_{\text{prev}} + \epsilon$ and inflates $\sigma_{\text{total}}$ by the magnitude of the adjustment, guaranteeing temporal causality $t_{\text{corrected}}(t_1) < t_{\text{corrected}}(t_2)$ for all $t_1 < t_2$.

ChronoTick also supports a conservative NTP-anchoring fallback mode that computes corrections relative to the most recent NTP measurement rather than the prediction cache, trading prediction precision for guaranteed accuracy anchored to an external time reference.

\subsection{Retrospective Correction}

The retrospective correction module integrates NTP measurements into the historical dataset, enabling the prediction pipeline to learn from errors and continuously improve.

Time series foundation models are autoregressive: they predict future values based on recent historical context. If the historical context contains uncorrected prediction errors, the model continues making similar mistakes. When high-confidence external synchronization occurs, the system must reconcile the discrepancy $\delta = s_{\text{true}} - \hat{s}$ between true measured skew $s_{\text{true}}$ and predicted skew $\hat{s}$ at time $t$. By replacing erroneous predictions with interpolated NTP ground truth, the corrected dataset teaches the model true clock behavior patterns. The production system employs backtracking correction, which provides the strongest learning signal by replacing the entire model context window with interpolated NTP measurements:

\begin{algorithm}[H]
\caption{Backtracking Retrospective Correction}
\label{algo:backtracking}
\begin{algorithmic}[1]
\Require Previous NTP skew $s_{\text{prev}}$ at time $t_{\text{prev}}$, Current NTP skew $s_{\text{curr}}$ at time $t_{\text{curr}}$, Historical timestamps $\{t_i\}_{i=1}^{n}$ in $(t_{\text{prev}}, t_{\text{curr}})$, Predicted skews $\{\hat{s}_{t_i}\}_{i=1}^n$, Context window size $w_{\text{context}}$
\Ensure Corrected skews $\{\hat{s}'_{t_i}\}_{i=1}^{n}$
\State $t_{\text{start}} \gets \max(t_{\text{prev}} - w_{\text{context}}, t_{\text{experiment\_start}})$ \Comment{Extend correction window}
\For{$i = 1$ to $n$}
    \If{$t_i \geq t_{\text{start}}$}
        \State $\alpha \gets \frac{t_i - t_{\text{prev}}}{t_{\text{curr}} - t_{\text{prev}}}$ \Comment{Linear interpolation}
        \State $\hat{s}'_{t_i} \gets s_{\text{prev}} + \alpha \cdot (s_{\text{curr}} - s_{\text{prev}})$
    \Else
        \State $\hat{s}'_{t_i} \gets \hat{s}_{t_i}$ \Comment{Preserve predictions outside window}
    \EndIf
\EndFor
\State \Return $\{\hat{s}'_{t_i}\}_{i=1}^{n}$
\end{algorithmic}
\end{algorithm}

The backtracking algorithm extends the correction window to cover the full context window that foundation models consume. For a model with context window $w_{\text{context}} = 512$ measurements at 1 Hz sampling, the correction window extends 512 seconds backward from the current NTP measurement. All predictions within this window are corrected with linear interpolation between NTP anchor points. This eliminates systematic biases, ensures the entire context window contains NTP-aligned data, and creates a continuous improvement loop where the model adapts to crystal aging and environmental changes without manual recalibration.

\subsection{Design Implications}

\textbf{Defense mechanisms.} ChronoTick implements multi-layer defensive validation ensuring robust operation despite noisy measurements and occasional model errors. External defenses protect against measurement anomalies: outlier filtering (adaptive EMA baseline with z-score rejection) and quality thresholds eliminate noisy NTP measurements before they contaminate the prediction pipeline. Internal defenses constrain model predictions to physically plausible ranges; predictions exceeding validation bounds are capped with inflated uncertainty, preventing erroneous forecasts from corrupting corrections. Monotonicity enforcement (Algorithm~\ref{algo:time_correction}) prevents backward time jumps when the prediction cache updates.

\textbf{Model independence.} The architecture is model-agnostic, requiring only (i) multi-step forecasting, (ii) probabilistic uncertainty outputs, and (iii) autoregressive context consumption. The current implementation uses TimesFM; the architecture supports substitution as foundation model research advances without requiring changes to the synchronization framework.

\textbf{Chrony integration.} Optional integration with chrony via its tempcomp directive~\cite{ntp_performance_chrony_ntpd} enables deployment flexibility. ChronoTick feeds supersampled drift predictions where chrony expects temperature compensation, enabling kernel-level clock discipline using foundation model forecasts. This represents a deployment enhancement, not a requirement; ChronoTick operates independently with direct NTP access.

\textbf{Time bounding.} Corrected timestamps include uncertainty bounds $\sigma_{\text{total}}$, supporting TrueTime-like intervals $[t_{\text{corrected}} - \sigma_{\text{total}}, t_{\text{corrected}} + \sigma_{\text{total}}]$~\cite{spanner_osdi2012_truetime_gps_atomic}. This enables distributed coordination algorithms to reason about temporal uncertainty: applications can trade latency against certainty by waiting for tighter bounds. The evaluation demonstrates 94.9\% coverage within stated bounds, meeting probabilistic guarantee requirements.


\section{Evaluation}

We evaluate ChronoTick through two complementary phases. First, we benchmark foundation models on clock drift data and validate architectural design choices through component ablation studies. This internal validation establishes the optimal system configuration: dual-model architecture with retrospective correction achieving 0.604 ms mean absolute error (MAE), representing 2.7$\times$ improvement over single-model alternatives.

Second, we demonstrate production performance across real-world scenarios: microsecond-scale client access latency validates deployment viability, defensive mechanisms maintain accuracy under measurement anomalies (outlier detection improving MAE by 74\%), unsynchronized clock experiments demonstrate bounded predictions even with natural drift rates reaching -7.877 ms/hour, and sustained 8-hour deployments achieve 94.9\% uncertainty bound coverage while maintaining 2.4-3.3$\times$ better temporal consistency than commodity NTP implementations. Experiments span three diverse platforms (consumer workstations running Windows, cloud servers running Debian, HPC cluster nodes running Ubuntu), validating zero-shot generalization across varying hardware configurations, operating systems, and network characteristics without requiring per-device training or calibration.

\subsection{Experimental Setup}

\textbf{Software.} All evaluations employ TimesFM foundation models (v2.5-200m) for temporal prediction. The system uses Python 3.9-3.12 across platforms with pytorch 2.1.0, timesfm 1.0.0, and numpy 1.24.0. NTP measurements are obtained using ntplib 0.4.0 querying pool.ntp.org servers. The production configuration implements dual-model architecture with inverse variance fusion and retrospective correction via backtracking. Validation methodology makes use of similar external NTP servers for ChronoTick and Chrony (pool.ntp.org), and relies on a separate set of servers (google.ntp.org) to acquire ground truth measurements every minute for comparison~\cite{ntp}. Runs span 1-8 hours collecting thousands of samples. For synchronized platform experiments, we compare against chrony 4.0-4.5~\cite{ntp_performance_chrony_ntpd} as the baseline NTP implementation.

\textbf{Sensors.} ChronoTick monitors hardware sensor streams via Linux hwmon, procfs, and sysfs interfaces: CPU core and package temperatures, operating frequencies, idle state (C-state) residency, memory pressure, I/O throughput, and system load. The exact sensor set varies by platform and privilege level (67--138 features across our testbed machines). Sensor readings are sampled alongside each NTP measurement and provided as multivariate input channels to the foundation models. On the Workstation platform (Windows), equivalent readings are obtained through platform-specific monitoring interfaces.

\textbf{Hardware.} Experiments utilize three distinct platforms representing diverse deployment scenarios:

\textbf{Workstation:} Consumer workstation running Windows with AMD Ryzen 5 3600, 16GB DDR4 RAM, and AMD Radeon 6950XT GPU.

\textbf{Cloud Media Server:} Dedicated server running Debian with Intel Core i7-6700 and 16GB DDR4 RAM.

\textbf{HPC Cluster:} Compute nodes running Ubuntu with dual Intel Xeon Silver 4114 processors and 46GB DDR4 RAM per node. Network connectivity through centralized proxy infrastructure.

\subsection{Foundation Model Benchmarking}

\begin{figure*}[h!]
    \centering
    \begin{subfigure}[b]{0.22\textwidth}
        \centering
        \includegraphics[width=\textwidth, trim=0 0 120 12, clip]{figures/MAE/unsy_cpu_short.png}
        \caption{Short Window CPU}
        \label{subfig:cpu_short}
    \end{subfigure}%
    \begin{subfigure}[b]{0.22\textwidth}
        \centering
        \includegraphics[width=\textwidth, trim=0 0 120 13, clip]{figures/MAE/unsy_cpu_long.png}
        \caption{Long Window CPU}
        \label{subfig:cpu_long}
    \end{subfigure}%
    \begin{subfigure}[b]{0.22\textwidth}
        \centering
        \includegraphics[width=\textwidth, trim=0 0 120 23, clip]{figures/MAE/unsy_gpu_short.png}
        \caption{Short Window GPU}
        \label{subfig:gpu_short}
    \end{subfigure}%
    \begin{subfigure}[b]{0.30\textwidth}
        \centering
        \includegraphics[width=\textwidth, trim=0 0 10 24, clip]{figures/MAE/unsy_gpu_long.png}
        \caption{Long Window GPU}
        \label{subfig:gpu_long}
    \end{subfigure}
    \caption{Foundation model MAE comparison for clock drift prediction across hardware platforms and prediction horizons. Data collected on Chameleon Cloud testbed over 24 hours.}
    \label{fig:mae-comparison}
\end{figure*}

To validate foundation model feasibility for drift prediction, we benchmarked selected models on 24-hour data collected from the Chameleon Cloud testbed~\cite{chamelon_citation}. Figure~\ref{fig:mae-comparison} compares models across short-window (5s, 10s) and long-window (20s, 40s, 60s) horizons on both CPU and GPU. For each model, we compute and average the median absolute error and execution time across 25 randomly selected but consistent samples, evaluating eight context window lengths ranging from 10s to 5 minutes.

The Chronos family consistently achieves the lowest MAEs for short horizons, with Chronos-mini delivering optimal accuracy (MAE of $1.22\times10^{-5}$) and speed (0.0168\,s GPU). TimesFM exhibits higher error ($6.39\times10^{-4}$) but provides native quantile outputs critical for uncertainty quantification. For long horizons, Chronos-Base achieves the lowest MAE of $1.55\times10^{-5}$, followed by Chronos-Small ($1.92\times10^{-5}$) and Chronos-Mini ($2.14\times10^{-5}$). We selected TimesFM as ChronoTick's primary engine for its native quantile capability and production-grade performance, recognizing that probabilistic outputs are essential for uncertainty quantification, a capability that polynomial approaches~\cite{graham_nsdi22} do not provide.

\subsection{Component Ablation Study}

\begin{figure*}[!t]
    \centering
    \includegraphics[width=\textwidth]{figures/2_mae_grouped.pdf}
    \caption{Comprehensive ablation study comparing alternative architectural choices across four design dimensions: model architecture, retrospective correction algorithms, drift rate estimation methods, and baseline smoothing.}
    \label{fig:design_comparison}
\end{figure*}

Before comparing ChronoTick against external baselines, we validate our architectural design choices through systematic ablation. We systematically disable or replace individual components to quantify their contributions to prediction accuracy, establishing the optimal configuration for production deployment. Each configuration runs for 1 hour on the Workstation platform, with mean absolute error between predicted and true clock offsets serving as the primary accuracy metric.

Figure~\ref{fig:design_comparison} compares alternative architectural choices across four design dimensions: model architecture (single minutes-horizon model, single hours-horizon model, or dual-model fusion), retrospective correction algorithms (none, linear interpolation, proportional adjustment, or backtracking), drift rate estimation methods (consecutive measurement differencing versus windowed linear regression over recent predictions), and baseline smoothing (exponential moving average filtering versus no smoothing).

The production configuration (dual-model architecture with backtracking retrospective correction, windowed drift rate estimation, and baseline smoothing) achieves 0.604 ms MAE, establishing this as the optimal configuration. Single-model architectures demonstrate fundamental limitations: the minutes-horizon model alone reaches 1.625 ms MAE (2.7$\times$ worse than production), while the hours-horizon model alone achieves 1.030 ms MAE (1.7$\times$ worse), validating the dual-model fusion design for combining immediate responsiveness with long-range forecasting capability.

Among dual-model variants, retrospective correction algorithms show substantial impact. Backtracking correction (0.604 ms) outperforms proportional adjustment (1.048 ms, 1.7$\times$ worse), linear interpolation (1.271 ms, 2.1$\times$ worse), and no correction (1.296 ms, 2.1$\times$ worse). The windowed drift rate estimation method reduces error from 1.117 ms to 0.604 ms (1.8$\times$ improvement). Baseline smoothing through exponential moving average filtering provides additional refinement, reducing error from 1.157 ms to 0.604 ms (1.9$\times$ improvement).

\subsection{Multivariate Sensor Contribution}

\begin{figure}[t]
    \centering
    \includegraphics[width=0.95\linewidth]{figures/sensor_shap_heatmap.png}
    \caption{SHAP feature importance across sensor categories and hardware platforms for clock drift prediction, computed over 24 hours of continuous data collection. Non-temperature categories contribute 82--85\% of total importance across all machines.}
    \label{fig:sensor_shap}
\end{figure}

To isolate individual sensor contributions, we collect 24 hours of continuous sensor and clock drift data from four machines (two commodity, two HPC; idle and stressed conditions), train gradient boosted regression models on clock drift rate using categorized sensor streams, and apply SHAP analysis~\cite{shap_nips2017} to quantify each category's importance. Figure~\ref{fig:sensor_shap} shows that non-temperature sensor categories dominate prediction importance on every machine, contributing 82--85\% of total SHAP importance. Full multivariate models achieve $R^2$ = 0.67--0.94 versus 0.47--0.78 for temperature-only models, confirming that temperature alone is insufficient for accurate drift prediction.

The dominant non-temperature category varies by hardware and workload: memory pressure on idle systems (34--46\%), CPU frequency under stress (42--59\%), with no single sensor category universally optimal. This hardware-dependent variation validates the multivariate architecture that integrates diverse hardware monitoring streams rather than relying on temperature alone.

\subsection{Client Access Performance}

\begin{figure}[!htbp]
    \centering
    \includegraphics[width=0.95\linewidth, trim=0 0 0 23, clip]{figures/1_access_performance.pdf}
    \caption{Performance comparison of ChronoTick's lock-free shared memory IPC architecture against direct NTP queries and native system clock access.}
    \label{fig:access_performance}
\end{figure}

The lock-free IPC shared memory architecture enables microsecond-level client access latency, validating practical deployment viability for latency-sensitive applications. As shown in Figure~\ref{fig:access_performance}, single-client access completes in 2.00 $\mu$s (orders of magnitude faster than direct NTP queries at 42.93 ms) while approaching native system clock access (0.11 $\mu$s). Concurrent access scales linearly from 2.00 $\mu$s (1 client) to 5.50 $\mu$s (8 clients), demonstrating the architecture supports parallel queries without contention or performance degradation.

\subsection{Defensive Mechanisms}


\begin{figure}[t]
    \centering
    \includegraphics[width=0.95\linewidth]{figures/5a_external_defense.pdf}
    \caption{External defense: outlier detection in HPC cluster environment with naturally occurring anomalies from proxy architecture.}
    \label{fig:external_defense}
\end{figure}

\begin{figure}[t]
    \centering
    \includegraphics[width=0.95\linewidth]{figures/5b_internal_defense.pdf}
    \caption{Internal defense: resilience under system clock chaos from conflicting dual synchronizers running simultaneously.}
    \label{fig:internal_defense}
\end{figure}

ChronoTick defends against two threat vectors: external corruption of input measurements and internal unreliability of model predictions. We evaluate external defense within the HPC cluster, which exhibits naturally occurring outliers from its proxy architecture where all nodes access external NTP servers through centralized gateway nodes. ChronoTick employs modified z-score outlier detection (threshold 3.0) to identify and filter anomalous measurements. For internal defense, we evaluate the Workstation under induced system clock chaos from conflicting dual synchronizers (Hyper-V TimeSync and Chrony) running simultaneously for 2 hours, creating erratic training signals that emulate scenarios where model predictions become unreliable.

Figure~\ref{fig:external_defense} shows ChronoTick detecting 34 outliers (14.3\% of samples, 420 ms maximum magnitude). With outlier filtering enabled, ChronoTick achieves 1.030 ms MAE and 1.449 ms RMSE. Without filtering (right, hypothetical), performance degrades to 3.903 ms MAE and 27.671 ms RMSE; a 74\% MAE improvement from defensive filtering.

Figure~\ref{fig:internal_defense} demonstrates Workstation resilience under adversarial conditions. During the 2-hour adversarial period (red shaded), system clock chaos exhibits 332 ms mean error and 231 ms standard deviation (ranging 50-820 ms). ChronoTick maintains 221 ms MAE despite erratic training signals from the conflicting synchronizers. After disabling the adversarial system (purple dashed line), the system requires approximately 2 hours to fully restore accuracy (green shaded recovery period) as retrospective correction gradually rebuilds clean historical context from valid NTP measurements, demonstrating the mechanism's self-healing capability.

These results validate multi-layer defensive mechanisms that maintain temporal consistency despite hostile conditions.

\subsection{Holdover Performance}


\begin{figure}[t]
    \centering
    \includegraphics[width=0.95\linewidth, trim=0 0 0 23, clip]{figures/exp13_homelab_unsync.pdf}
    \caption{Validation of ChronoTick's predictive capability on unsynchronized Cloud Media Server exhibiting natural clock drift by transitioning from 8-hour unsynchronized operation to synchronized deployment with system NTP disabled.}
    \label{fig:unsync_test}
\end{figure}

Before evaluating sustained synchronized deployments, we validate ChronoTick's predictive capability on a clock exhibiting natural drift by transitioning from unsynchronized to synchronized operation. This experiment demonstrates the system's ability to track clock behavior and provide bounded error even without valid external synchronization, a fundamental requirement for reducing synchronization frequency in resource-constrained environments.

Figure~\ref{fig:unsync_test} presents an 8-hour unsynchronized deployment on the Cloud Media Server with system NTP disabled. The system clock diverges at -7.877 ms/hour (a natural drift rate typical of commodity hardware), and ChronoTick maintains 16.022 ms MAE throughout the experiment, demonstrating that foundation models can extract sufficient temporal structure for bounded predictions without requiring well-behaved synchronized clocks. This critical holdover scenario validates drift prediction on free-running oscillators.


\subsection{Sustained Production Deployments}


\begin{figure*}[!t]
    \centering
    \begin{subfigure}[b]{0.48\textwidth}
        \centering
        \includegraphics[width=\textwidth, trim=0 0 0 20, clip]{figures/longterm_node1_offset.pdf}
        \caption{Node 1 offset behavior}
        \label{subfig:longterm_node1_offset}
    \end{subfigure}%
    \hfill
    \begin{subfigure}[b]{0.48\textwidth}
        \centering
        \includegraphics[width=\textwidth, trim=0 0 0 20, clip]{figures/longterm_node1_cumulative.pdf}
        \caption{Node 1 cumulative error}
        \label{subfig:longterm_node1_cum}
    \end{subfigure}\\[-0.5em]
    \begin{subfigure}[b]{0.48\textwidth}
        \centering
        \includegraphics[width=\textwidth, trim=0 0 0 20, clip]{figures/longterm_node2_offset.pdf}
        \caption{Node 2 offset behavior}
        \label{subfig:longterm_node2_offset}
    \end{subfigure}%
    \hfill
    \begin{subfigure}[b]{0.48\textwidth}
        \centering
        \includegraphics[width=\textwidth, trim=0 0 0 20, clip]{figures/longterm_node2_cumulative.pdf}
        \caption{Node 2 cumulative error}
        \label{subfig:longterm_node2_cum}
    \end{subfigure}
    \caption{Sustained 8-hour production deployments on HPC cluster nodes: offset behavior and cumulative error against chrony for both Node 1 and Node 2.}
    \label{fig:longterm_sustained}
\end{figure*}

Having validated incremental design decisions, defensive mechanisms, and unsynchronized performance, we now evaluate end-to-end production capability through sustained 8-hour deployments on synchronized systems. These experiments demonstrate ChronoTick's ability to supersample clock behavior: using sparse NTP measurements combined with foundation model predictions to achieve sub-millisecond precision between synchronization points.

Figure~\ref{fig:longterm_sustained} presents results from two HPC cluster compute nodes over 8-hour continuous deployments. Both nodes maintain synchronized clocks via chrony while ChronoTick provides predictive corrections between NTP measurements (arriving every minute). Node 1 achieves 0.91 ms MAE while Node 2 achieves 0.49 ms MAE, demonstrating sub-millisecond precision through the dual-model architecture combining minutes-horizon and hours-horizon forecasting.

The cumulative error plots show bounded error accumulation throughout 8-hour deployments and report chrony's system clock MAE alongside ChronoTick's: Node 1 achieves 0.91 ms versus chrony's 2.18 ms (2.4$\times$ improvement), while Node 2 achieves 0.49 ms versus chrony's 1.60 ms (3.3$\times$ improvement). Across both nodes, the system successfully bounds 94.9\% of all predictions within stated confidence intervals through per-prediction uncertainty quantification, approaching TrueTime-style probabilistic guarantees for distributed temporal reasoning.

These sustained deployments validate the complete system: zero-shot foundation models generalize across heterogeneous platforms without per-device calibration, retrospective correction sustains sub-millisecond accuracy through continuous refinement, and multivariate sensor integration provides environmental context enabling adaptive prediction.

\section{Conclusions}

ChronoTick demonstrates that pre-trained time series foundation models can achieve sub-millisecond clock accuracy on commodity hardware without per-device calibration or specialized infrastructure. The system makes three key contributions. First, zero-shot inference eliminates the calibration barrier entirely, enabling immediate deployment across heterogeneous platforms without training data collection or model fitting per device. Second, retrospective correction continuously refines predictions as ground truth from synchronization events becomes available, adapting to changing thermal conditions, workload patterns, and oscillator aging without manual recalibration. Third, multivariate sensor integration captures joint thermal, frequency, and workload dynamics that single-sensor polynomial approaches cannot represent. Across diverse platforms spanning consumer workstations, cloud servers, and HPC clusters, ChronoTick achieves 2.4-3.3$\times$ better temporal consistency than commodity NTP implementations through learned prediction on commodity sensor streams.

Several directions extend this work. GPS PPS integration could provide a low-cost stratum reduction path; a single pulse-per-second signal combined with ChronoTick's predictive capability would maintain accuracy between pulses through the learned drift model. The existing uncertainty quantification framework, which achieves 94.9\% coverage in sustained deployments, provides a foundation for deriving formal clock bound guarantees. Looking further, this uncertainty-aware architecture paves the way toward probabilistic time bounds similar to TrueTime, enabling distributed coordination protocols to reason explicitly about clock error rather than assuming worst-case drift rates. Together, these directions suggest that foundation model-based compensation can close the gap between commodity clocks and precision timing infrastructure.


\bibliographystyle{IEEEtran}
\bibliography{chronotick}

\end{document}
