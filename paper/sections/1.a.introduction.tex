\section{Introduction}

The growing demands of distributed applications have driven a need for tightly synchronized clocks. Instead of coordinating over the network, applications can establish event ordering using timestamps from local clocks, but only if those clocks are sufficiently accurate. From distributed databases~\cite{cockroachdb_hlc_implementation} and tracing systems~\cite{dapper_google2010} to log stores~\cite{chronolog_msst2020, xplog_tsc2026}, scientific computing workflows~\cite{denis_starpu_tracing2023}, or agentic AI system observability, where the community is moving towards chains of thousands of asynchronous autonomous events across process boundaries~\cite{agentsight_pacmi2025}. When clocks disagree, events appear out of order, causal chains break, and system behavior becomes difficult to diagnose or reproduce. Achieving sub-millisecond accuracy has traditionally required specialized infrastructure such as PTP-capable network interfaces, GPS receivers, or atomic references~\cite{ptp}, yet commodity servers, edge deployments, and heterogeneous clusters lack access to such hardware.

Software-based compensation of the local oscillator offers an alternative path: correcting drift using sensors already present on commodity motherboards. Between synchronization events, commodity servers free-run on local quartz oscillators whose frequency varies with temperature, supply voltage, and aging~\cite{tirado_andres_clock_sources_2019}. NTP~\cite{ntp} periodically corrects clock offset over the network, but drift accumulates during holdover. Hardware solutions such as PTP-synchronized networks, GPS-disciplined references, and atomic clocks can achieve nanosecond-level accuracy; datacenter systems such as Sundial~\cite{sundial_osdi2020} and Huygens~\cite{huygens_nsdi2018_geng_svm_nanosec} demonstrate what becomes possible with dedicated timing infrastructure. For the vast majority of commodity deployments where such infrastructure is impractical, software compensation has emerged as a practical alternative. Chrony's \texttt{tempcomp} directive~\cite{ntp} applies a static polynomial to a single temperature reading. Graham~\cite{graham_nsdi22} reported that automatic cubic polynomial fitting to DIMM temperature readings could reduce commodity oscillator drift by up to three orders of magnitude. Across adjacent domains, from satellite clock bias prediction~\cite{he_bds3_lstm_72pct_arima, gnss_clock_prediction_review_2025} to TCXO frequency compensation~\cite{su_lstm_tcxo_transfer_learning}, machine learning has consistently improved over polynomial baselines.

Yet existing software compensation approaches share limitations that constrain practical deployment. Per-device calibration remains a barrier, with current systems requiring extended learning periods per server before compensation can begin~\cite{graham_nsdi22, su_lstm_tcxo_transfer_learning}. Current methods are designed for single-variable correction, typically relying on a single temperature sensor without addressing the joint dynamics of other node variables such as CPU frequency, memory pressure, idle states, and I/O load~\cite{tirado_andres_clock_sources_2019}. Once the compensation model is fit, it remains fixed; thermal conditions, workload patterns, and environmental factors shift during operation~\cite{tirado_andres_clock_sources_2019}, yet no existing approach provides mechanisms for continuous self-improvement or retrospective correction. Finally, state-of-the-art solutions are either closed-source~\cite{graham_nsdi22} or tied to proprietary infrastructure~\cite{huygens_nsdi2018_geng_svm_nanosec, sundial_osdi2020}; there is a clear need for an accessible precision software clock on commodity hardware~\cite{najafi_hotos2021_time}.

Time series foundation models offer a new capability for addressing these gaps. Pre-trained on large and diverse corpora, models such as Chronos~\cite{Chronos}, TimesFM~\cite{TimesFM}, MOIRAI~\cite{MOIRAI}, and MOMENT~\cite{MOMENT} learn generalizable temporal patterns that transfer to unseen domains without per-device training or fine-tuning. Models such as TimesFM and MOIRAI natively support multivariate input, jointly modeling cross-variable dynamics, while others such as Chronos operate on individual channels. PatchTST~\cite{PatchTST} achieves strong transfer performance through patch-based architecture, and Time-MoE~\cite{Time-MoE} demonstrates billion-parameter scale, illustrating the breadth and maturity of the field. These models provide probabilistic prediction intervals alongside point forecasts. Rather than static polynomial fitting, they operate through continuous inference on streaming data, enabling a shift from per-device calibration to adaptive learned prediction that generalizes across hardware. Open pre-trained weights make such systems reproducible and accessible. To the best of our knowledge, no prior work has applied time series foundation models to oscillator drift prediction on commodity hardware.

We present ChronoTick, a system that applies zero-shot time series foundation models to predict and correct oscillator drift on commodity hardware, requiring no per-device calibration and no specialized infrastructure. ChronoTick runs multiple foundation models in parallel, fusing their predictions and refining estimates retrospectively as ground truth from synchronization events becomes available. It consumes multivariate input from Linux hwmon sensor streams, including thermal zones, CPU frequency, and system load, and integrates directly with chrony for real-time NTP correction. We release ChronoTick as open source. Across heterogeneous platforms and without per-device training, ChronoTick achieves sub-millisecond accuracy (0.49--0.91\,ms mean absolute error) and 2.4--3.3$\times$ better offset stability than commodity NTP.

This paper makes the following contributions:
\begin{enumerate}
\item \textbf{Zero-shot foundation model drift prediction.} To our knowledge, the first application of pre-trained time series foundation models to commodity oscillator compensation, eliminating per-device calibration entirely.
\item \textbf{Self-improving prediction pipeline.} A pipeline that retrospectively refines drift estimates as ground truth from synchronization events becomes available, enabling continuous adaptation to changing conditions without manual recalibration.
\item \textbf{Multivariate compensation from commodity sensors.} Integration of multiple hardware monitoring streams to capture the joint thermal, frequency, and workload dynamics that single-sensor approaches miss.
\end{enumerate}
