\section{Conclusions}

ChronoTick demonstrates that pre-trained time series foundation models can achieve sub-millisecond clock accuracy on commodity hardware without per-device calibration or specialized infrastructure. The system makes three key contributions. First, zero-shot inference eliminates the calibration barrier entirely, enabling immediate deployment across heterogeneous platforms without training data collection or model fitting per device. Second, retrospective correction continuously refines predictions as ground truth from synchronization events becomes available, adapting to changing thermal conditions, workload patterns, and oscillator aging without manual recalibration. Third, multivariate sensor integration captures joint thermal, frequency, and workload dynamics that single-sensor polynomial approaches cannot represent. Across diverse platforms spanning consumer workstations, cloud servers, and HPC clusters, ChronoTick achieves 2.4-3.3$\times$ better temporal consistency than commodity NTP implementations through learned prediction on commodity sensor streams.

Several directions extend this work. GPS PPS integration could provide a low-cost stratum reduction path; a single pulse-per-second signal combined with ChronoTick's predictive capability would maintain accuracy between pulses through the learned drift model. The existing uncertainty quantification framework, which achieves 94.9\% coverage in sustained deployments, provides a foundation for deriving formal clock bound guarantees. Looking further, this uncertainty-aware architecture paves the way toward probabilistic time bounds similar to TrueTime, enabling distributed coordination protocols to reason explicitly about clock error rather than assuming worst-case drift rates. Together, these directions suggest that foundation model-based compensation can close the gap between commodity clocks and precision timing infrastructure.
